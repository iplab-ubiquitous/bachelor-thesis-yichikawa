\chapter{実装}\chaLabel{implementation}
\section{概要}
\SysName は,ハードウェアとして三角法を用いた光学式距離センサ10個を一列に並べたセンサアレイと,センサから取得した値の処理と値を元に座標を計算するソフトウェアからなる.
\section{プロトタイプ} 
\secLabel{proto1}
\subsection{ハードウェア}
距離センサはSHARP GP2Y0E03\footnote{http://www.sharp.co.jp/products/device/lineup/selection/opto/haca/diagram2.html}を使用した.この距離センサは三角測量の原理を用い,対象までの距離を計測する.個々のセンサは,スレーブアドレスが初期値(0x40)で統一されているために,アプリケーションノート\footnote{http://www.sharp.co.jp/products/device/doc/opto/gp2y0e02\_03\_appl\_j.pdf}に記載されているe-fuseプログラミングの手順で,スレーブアドレスの変更を行なっている.
\SysName では,長さ約30\si{cm}のプラスティック製の定規に、両面テープでセンサ本体を30\si{mm}間隔で定規に固定し,配線類はセロハンテープで固定した.
%30\si{mm}という間隔は、センサアレイが膝が左右に動く範囲を全てカバーできる範囲として設定した。
\refImg{im1}は実際に製作したプロトタイプの一部である.
距離センサは横向きにして1列に並べた.これは、GP2Y0E03のアプリケーションノートには,センサの設置方向は,移動物体の距離を測定する時、\refImg{im2}の黒い矢印の方向ではなく,赤い矢印の方向で移動した方が、誤差が少ないとあるためである.
\par
個々の距離センサはArduino MEGA 2560と\iic で接続される。接続はユニバーサル基板上で行う。各センサの電源、グランド、\iic のクロック線、\iic のデータ線同士を、それぞれ基板の裏側の導線と接続する。Arduino MEGA 2560は、PCとUSBシリアル通信で接続される。




\subsection{ソフトウェア} 
Arduinoでは、\iic 通信でセンサからそれぞれの膝との距離を取得し、USBシリアル通信で10個を1フレームとした距離データを送信する。PC上のプログラムでは、受信した距離データからカーソルの座標を計算し、マウスカーソルを描画する。プログラム言語はPythonを用いた.シリアル通信のためのライブラリとしてPySerial,ポインタを描画するGUIのためのライブラリとしてPyQtを用いた.
膝の位置の計算には,
%スマートウォッチに搭載した距離センサから指の位置をトラッキングした
Xiaoら\cite{Xiao:2018:LOP:3173574.3173669}の方法を参考にし、時間$t$におけるカーソルの座標$(C^t_x, C^t_y)$を次のように計算する.
\begin{enumerate}
	\item 各距離センサの値を指数平均平滑フィルタを用いて平滑化する.これを$D^t_i \ (0 \leq i \leq 9)$と表す。
	\item $C^t_y$を$D^t_i $の最小値とする.
		\begin{eqnarray}
		 	C^t_y = \min_i(D^t_i)
		 	\label{formula:f1}
		\end{eqnarray}
	\item $i$番目の距離センサについて,重み$w_i$を式\ref{formula:f2}のように計算する.ここで,$d$は重み調整の定数である.\SysName では調整の結果$d=2$としている.
		\begin{eqnarray}
			w_i = \cfrac{1}{D^t_i - C^t_y + d}
		\label{formula:f2}
	\end{eqnarray}
	\item $w_i$から,$C^t_x$を計算する.
		\begin{eqnarray}
		 	 C^t_x=\cfrac{\sum_i iw_i}{\sum_i w_i}
		 	\label{formula:f3}
		\end{eqnarray} 
	\item $(C^t_x, C^t_y)$を指数平均平滑フィルタを用いて平滑化する.
\end{enumerate}
\img{htbp}{1.0}{proto.pdf}{製作したプロトタイプの一部}{im1}

%\subsection{プロトタイプ1の性能評価}\subsecLabel{proto1_problem}
%実装を行なったプロトタイプを机の裏に設置し,自由なポインタの操作ができるか試みた.キャリブレーション次第では膝を静止させた時にポインタも静止するが,多くの場合は膝を静止させているにも関わらずポインタは左右に振れるなどした.センサからの値を観察したところ,\fixme{膝がかかっていない部分の値}が激しく上下していることがわかった.原因として,以下のようなものが挙げられた.

\img{htbp}{1}{proto2-1}{移動物体に対する距離センサの設置方向概念図}{im2}
%\section{プロトタイプ2}
%\subsection{改良点}
%\refSubsec{proto1_problem}であげた問題を解決するために,プロトタイプに改良を行った.
%\begin{itemize}
	%\item{改良1: } 問題点1.について,すべてのセンサを90度左回転させて設置した.また同時に問題点4.について,
	%\item{改良2: } 問題点2.について,実装を\fixme{ブレッドボードでの接続からユニバーサル基板を用いて配線を固定すること}とした.
	%\item{改良3: } 問題点3.について,センサの直下には床に白紙を貼り付けることで解決を図った.
%\end{itemize}

%\subsection{改良後の評価}
%改良1を加えたプロトタイプ2を同様に机の裏に設置し,自由なポインタの操作ができるか実験した.プロトタイプ1ではポインタの左右の振れ方はとても大きく,キャリブレーション次第では操作もままならないほどであったが,改良1により軽減された.問題点1.の指摘は正しく,改良後のセンサの配置方向が正しいと考えられる.しかし,依然としてポインタが左右に振れた.
%\par
%改良2を加えて同様の実験を行なったところ,ポインタの左右の振れの改善はあまり見られなかった.これは,\fixme{センサと膝は直接触れることはないため,設置した状態から動くことはないからであると考える.}しかし,ユニバーサル基盤上に実装したことで,接続部分がブレッドボードよりも薄くなり机裏への設置が容易になった.加えてプロトタイプ1では作業中に簡単に配線が抜けてしまうことがあったが,配線が固定され抜けることがなくなった.したがって,ブレッドボードの実装に戻すことはしなかった.
%\par
%改良3を加えて,まずセンサの値を観察したところ,相変わらずノイズは観察されるが,指数平均平滑フィルタを通した後の値は改良3を加える前よりも安定した.床面を白くすることで赤外線の反射が多くなり,正しく測定できているからであると考える.

