\chapter{関連研究}

\section{足をカーソル操作として用いる研究}
関連研究として、足を入力操作に用いる研究は、古くから行われている。Englishら\cite{1698228}は、テキスト選択においていくつかの膝を含めた装置やデバイスを用いた時の操作時間を調査した。調査の結果、膝による操作は最も短い時間で選択することができることがわかった。
近年でも、調査が行われている。Vellosoら\cite{velloso:hal-01599657}は、座っている状態の机の下の足の動きの特徴を調査した。この論文の中で、机の下に配置したトラッキングシステムから、片方の足のつま先をマウス操作に割り当て、1次元と2次元におけるポインティング作業により、パフォーマンスのテストを行なっている。
田中ら\cite{110004704997}は、足の指をマウス操作に用いるために、母指の力制御と運動特性を調査した。
Horodniczyら\cite{Horodniczy:2017:FHE:3025453.3025625}は、靴底に可変摩擦式の装置を取り付け、足をマウス操作に用いることを実現した。靴底には低摩擦材料と高摩擦材料の2つを取り付け、高摩擦材料の位置を制御することで、かかと部分の摩擦力を調整している。結果、2次元のポインティングタスクでエラー率においてマウスより優れた結果を発表した。
Alexanderら\cite{Alexander:2012:PYB:2207676.2208575}は、モバイルデバイスのコマンドに対し、足によるジェスチャをマッピングするためのユーザ導出型の調査を行なった。
Felberbaumら\cite{Felberbaum:2018:BUF:3173574.3173908}は、立った状態、座った状態、投影された画面の上にいる状態の3条件で、GUIに関する操作、仮想空間に関する操作の2種類に対するジェスチャマッピングを調査した。この調査の中ではまた、ジェスチャと操作の対応が一意的かを表す指標を導入し評価を行なった。
鈴木\cite{ssuzuki_2009}は、測域センサを用いることで、足の動きをセンシングし、床面におけるインタラクション手法を提案している。
奥村\cite{okumura_2011}は、靴に加速度と角速度を取得することができるセンサを取り付け、外出時におけるモバイル端末の操作を行うシステムを開発した。

