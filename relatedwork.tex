\chapter{関連研究}\chaLabel{relatedwork}
本章では,関連する研究について述べる.まず,現在広く行われている,足によるジェスチャ入力を行う研究について示す.次に本研究で想定する,机上のコンピュータの操作を想定した足を用いた入力の研究を示す.最後に,膝による入力を用いた研究を示す.
%1
\section{足をジェスチャ入力として用いる研究}
足を用いたコンピュータへの入力の目的はいくつか存在する.まず,屋外でスマートフォンなどを操作する時に荷物を持っている,手が汚れているというすぐに手を用いることができない状況を想定したものがある.
Alexanderら\cite{Alexander:2012:PYB:2207676.2208575}は,モバイル端末で頻繁に用いられる操作に対し,足ジェスチャを割り当てるための調査を行なった.	
Fanら\cite{Fan:2017:ESF:3123021.3123043}は,
%荷物を持っている,手が汚れているといった状態において,
足のジェスチャによりモバイル端末を操作することに対する実証研究を行なった.ユーザ定義の足のジェスチャを用いた方法と,荷物を降ろして手で端末を持ち操作する方法を比較したところ,前者の方が70\%高速な操作が可能であるという結果となった.
HanらのKick\cite{Han:2011:KIU:2037373.2037379}では,蹴り出すジェスチャを端末操作に用いるために,ユーザがキックの方向と速度をどの程度制御できるかを調査した.
奥村\cite{okumura_2011}は,靴に加速度と角速度を取得することができるセンサを取り付け,外出時におけるモバイル端末の操作を行うシステムを開発した.
本研究は,屋内での利用のみを想定した環境設置型の装置を用いるという点と,ラップトップコンピュータやデスクトップコンピュータへの利用を想定しているという点から,インタラクションの目的は異なる.

次に屋内環境に設置する装置を用いた研究事例を紹介する.
AugstenらによるMultitoe\cite{Augsten:2010:MHI:1866029.1866064}では,巨大なタッチパネルを床面に設置し,複数の足の認識や足の重心位置の認識した.これにより,床面に表示されたメニューやキーボードを足で操作することを提案した.
鈴木\cite{ssuzuki_2009}は,測域センサによって足の動きをセンシングし,床面におけるインタラクション手法を提案している.
これらの調査やインタラクション技術は立った状態を想定しているものである.本研究では,デスクトップ上での作業中という限定された環境におけるインタラクション手法の提案を目指す.

%2
\section{机上のコンピュータの操作を想定した足を用いた入力の研究}

本節では,机上におけるコンピュータの操作に足を用いた入力手法を調査,提案した研究を紹介する.
Felberbaumら\cite{Felberbaum:2018:BUF:3173574.3173908}は,立った状態,座った状態,投影された画面の上にいる状態の3条件で,GUIに関する操作,仮想空間に関する操作の2種類に対して,どの足ジェスチャを用いるのが好ましいかを,ユーザに対する調査で明らかにした.
Saundersら\cite{Saunders:2016:TFI:2901790.2901815}は,立った状態でのデスクトップアプリケーションの制御に足を用いた.
Pearsonら\cite{Pearson:1986:MMD:22627.22392, Pearson:1988:EET:49108.1046356}は「モル」という装置を開発し,ポインタの操作などに手の代わりに足を使用する方法を調査した.モルを用いた場合でも,訓練によって小さなターゲットを選択することが可能になることを示した.
Horodniczyら\cite{Horodniczy:2017:FHE:3025453.3025625}は,ユーザの靴に可変摩擦式の装置を取り付け,足によるカーソル操作の補助装置として用いた.靴底には低摩擦材と高摩擦材の2つを取り付け,高摩擦材の接地圧力をステッピングモータで制御する.足の位置をカメラにより取得し,ターゲットに近づくにつれ圧力を高める.
Vellosoら\cite{velloso:hal-01599657}は,座っている状態の机の下の足の動きの特徴を調査した.机の下に配置したカメラから,片方の足のつま先をマウス操作に割り当て,1次元と2次元におけるポインティング作業により,パフォーマンスのテストを行なっている.
田中ら\cite{110004704997}は,足の指をマウス操作に用いるために,母指の力制御と運動特性を調査した.
これらの研究では,大型の装置を用いているために持ち運びや設置が困難であったり,靴に対して装置を取り付けるためにユーザに身体上の制約を強いてしまう.
%また,これらの研究で用いられているのは足先であり,
本研究では小型で設置が簡単かつ足に装置を取り付けないアプローチで,問題の解決を図る.また,本研究は足先でなく膝に焦点を当てることで,既存手法との組み合わせによってさらなるインタラクションの拡張を図ることも可能である.

足と他の入力モダリティとの組み合わせを行なった研究という点では,次のようなものが存在する.
G\"{o}belら\cite{Gobel:2013:GFI:2468356.2479610}が提案する手法では視線と足を組み合わせ,視線位置におけるパンとズームの操作を足によるペダル操作で行うことを提案した.
Rajanna\cite{Rajanna:2016:GFI:2876456.2876462}は,視線によるポインティングと足によるクリックコマンドで構成されるシステムを構築した.
本研究では膝による入力操作を行うことで,足を用いた他の手法との組み合わせの可能性を探る.

\section{膝を入力手法として用いる研究}
\img{htbp}{0.7}{eng.png}{English\cite{1698228}らの膝操作手法}{engknee}
膝に関する研究の中で,コンピュータへの入力を想定したものは少ない.
Englishら\cite{1698228}は,テキスト選択において,膝を含めたいくつかの装置やデバイスを用いた時の操作時間を調査した.調査の結果,膝による操作は最も短い時間で選択することができることがわかった.
この論文では,机の下に取り付けた装置のレバーを膝で動かすことで入力を行なったが,装置が複雑であるという欠点がある(\refImg{engknee}).我々は単純な構造のプロトタイプを開発することで,この研究に貢献する.

%	\begin{itemize}
%		\item 足ジェスチャの研究が様々ある→机の下という環境を想定したものが少ない
%		\item 机の下という環境を想定した研究が少数だが存在する→先述の問題点がある,膝を用いれば組み合わせることができる
%		\item 膝を用いた研究はほぼない→装置が大型である,膝の研究が少ない
%	\end{itemize}



