\chapter{膝位置認識システム}\chaLabel{system}
\section{概要}
本研究では,机の下の片方の膝の位置を認識するシステムを提案する.システムの流れは以下のようになっている.
\begin{enumerate}
	\item ユーザは机の裏に距離センサアレイを設置し,自分のPCと接続する.その後机の前に座り,膝を上下左右に動かす.
	\item 距離センサアレイが,それぞれセンサと膝との距離を計測する.
	\item 全ての距離センサの計測値は膝位置計算ソフトウェアに送信される.
	\item ソフトウェアは,机に垂直な面において膝がどの位置にあるかを計算する.
	\item 膝の位置に応じてアプリケーションを実行する.
\end{enumerate}
この時ユーザは机の裏に距離センサアレイを設置し,PCと接続することのみを行い,膝には何も装着しない.
\section{利用イメージ}
\subsection{ワープロソフト利用時のマウスカーソル操作}
\subsecLabel{word}
ワープロソフトを利用する時,手はキーボード上にあることが多い.しかし,文字色や文字サイズの変更,図の挿入など,コマンド入力で行うことができない操作を行う時,マウスやタッチパッドに手を移動させ,操作する場面が多い.そこで,膝によってマウスカーソルを操作することで手の移動を削減することができる.
\img{htbp}{1.0}{usc1.pdf}{ワープロソフトの利用時に膝でマウスカーソルを操作するイメージ}{usc1}
\subsection{画面を見ながら作業するときの画面操作}
PCの画面を見ながら作業をする,例えば画面では回路図を写しながら,その回路図に従って電子回路を製作するときに,回路図をズームしたり別の回路図を見ようとすると,作業から手を離さなければいけない.そこで,膝を動かすことで手は作業に集中したまま画面の操作を行うことができる.\refImg{usc2}は
\img{htbp}{1.0}{usc2.pdf}{画面を見ながら作業をするイメージ}{usc2}
\section{本研究で行う操作}
本研究では膝の位置を認識することで,\refSubsec{word}で述べたようなマウスカーソル操作への応用を行う.これは,マウスカーソルのような繊細な操作を膝で行ったときの操作性や疲労感を明らかにし,膝による操作全体の改良を図るためである. 



