\chapter{膝の位置認識方法の設計}\chaLabel{system}
\section{概要}
本研究では,机の下の片方の膝の動きを認識することでマウスカーソルの操作を行うシステムを提案する.システムの流れは以下のようになっている.
\begin{enumerate}
	\item ユーザはシステムを設置した机の前に座る.
	\item ユーザが膝を動かすことができる範囲を記録するため,キャリブレーションを行う.
	\item ユーザは膝を上下左右に動かすことで,画面上のマウスカーソルが移動する.
\end{enumerate}
\section{認識する膝の動作}
%\subsection{操作方法案1}
%マウスの操作では,カーソルを$x$軸方向の操作にはマウスを左右に動かす方法,$y$軸方向の操作にはマウスを\fixme{奥に押す},または手前に引くという動作を行う.これらを膝に置き換えた時,
マウスの操作と異なり,前方や後方に動かすことはできない.したがって膝によるマウスカーソルの操作は,ユーザが膝を上下左右に動かすことによって行われる.\refImg{ov}は膝でマウスカーソルを操作するときのイメージである.
\img{htbp}{1}{sousa_overview.pdf}{膝によるマウスカーソル操作のイメージ}{ov}
ユーザはマウスカーソルを左右に移動させたい時には,膝を左右に傾ける.上方向に移動させたい時は,かかとを浮かせて膝を机に近づける.逆に下方向に移動させたい時は,足を手前に引き,その時に浮いたかかとを床に近づけることで,膝を机から遠ざける.\refImg{sousa3}は,上下方向にマウスカーソルを操作する時の,膝の操作のイメージである.
\img{htbp}{1}{sousa3.pdf}{上下方向の操作イメージ1}{sousa3}

同じ操作は\refImg{sousa1}に示されているような操作を行うことでも可能である.しかし,足が完全に地面についている時に,マウスカーソルは画面の一番下の位置になってしまう.そのため,ユーザが画面の真ん中付近にある\fixme{コンテンツ}を選択する時にかかとを浮かせた状態を維持しなければならず,疲労感が高くなってしまう.対して\refImg{sousa3}では,足が完全に地面についている時にカーソルが真ん中にあるため,ユーザは比較的楽な姿勢で操作することができる.

\img{htbp}{1}{sousa1.pdf}{上下方向の操作イメージ2}{sousa1}
%適用することができるが,膝を奥に動かしたり手前に引くということはできない.したがって,$y$軸方向の操作として,膝を持ち上げる動作を適用することとした.\refImg{sousa1}は$y$軸方向の操作方法として考案した膝の動かし方の1つである.画面の下の方へカーソルを動かす時には,かかとを下げ,カーソルが下限にくる時,かかとは完全に床につく状態になる.逆に画面の上の方へカーソルを動かす時には,かかとを浮かせる.


%しかし,この操作方法について意見を募ったところ,かかとを浮かせた状態で維持することが困難であり,著しく疲労を感じるという回答を受けた.特に,カーソルを画面の真ん中で維持することが困難であった.したがって,カーソルを画面の上下方向の真ん中に維持する時,膝の姿勢がユーザにとって特別な力を必要としない状態を取ることが望ましいことがわかった.
%\subsection{操作方法案2}
%操作方法案1の改善点を受け,操作方法案2を考えた.\refImg{sousa2}はそのイメージ図である.
%操作方法案2では足を手前に引き,かかとを浮かせた状態をカーソルが真ん中に来るものとする.カーソルを下に動かしたいときは,その位置からかかとを下げる.それに伴い,膝も下に下がる.カーソルを上に動かしたいときは操作方法案1と同様にかかとを更に浮かせる.
%操作方法案2についても意見を募った.しかし,カーソルを下方向に動かす動作は,足首が本来とは逆の方向に曲がるために,正しい操作が可能かはその人の足首の柔らかさに依存した.したがって,\fixme{足首に負担がかからない動作方法}について,操作方法案を再考することとした.
%\subsection{操作方法案3}
%操作方法案2の改善点を受け,操作方法案3を考えた.\refImg{sousa3}はそのイメージ図である.
%操作方法案3では,かかと含め,足が完全に床についている状態をカーソルが画面の真ん中に来るものとする.カーソルを下に動かしたいときは,足を十分引き膝の位置を下げる.カーソルを上に動かしたいときは操作方法案1と同様にかかとを浮かせる.
%操作方法案3について意見を募ったところ,操作方法案1で問題になった疲労感,操作方法案2で問題となった足首に対する依存が解消されたとの回答を受けた.したがって,この操作方法でカーソル操作を行うものとする.


