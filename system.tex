\chapter{システム}
\section{概要}
本研究では、机の下の片方の膝の動きを認識することでマウスポインタの操作を行うシステムを提案する。システムの流れは以下のようになっている。
\begin{enumerate}
	\item ユーザはシステムを設置した机の前に座る。
	\item ユーザが膝を動かすことができる範囲を記録するため、キャリブレーションを行う。
	\item センサから値を読み取り、膝の位置を2次元座標で表現する
	\item キャリブレーション時に記録した値を元に、画面のサイズに合わせて膝の位置をマップする。
\end{enumerate}
\section{膝の操作}
\subsection{操作方法案1}
カーソルの操作を行う時の膝の動作について考える。マウスの操作では、ポインタを$x$軸方向の操作にはマウスを左右に動かす方法、$y$軸方向の操作にはマウスを\fixme{奥に押す}、または手前に引くという動作を行う。これらを膝に置き換えた時、$x$軸方向の操作には膝を左右に傾けるという動作を適用することができるが、膝を奥に動かしたり手前に引くということはできない。したがって、$y$軸方向の操作として、膝を持ち上げる動作を適用することとした。\refImg{sousa1}は$y$軸方向の操作方法として考案した膝の動かし方の1つである。画面の下の方へポインタを動かす時には、かかとを下げ、ポインタが下限にくる時、かかとは完全に床につく状態になる。逆に画面の上の方へポインタを動かす時には、かかとを浮かせる。


しかし、この操作方法について意見を募ったところ、かかとを浮かせた状態で維持することが困難であり、著しく疲労を感じるという回答を受けた。特に、ポインタを画面の真ん中で維持することが困難であった。したがって、ポインタを画面の真ん中に維持する時、膝の姿勢がユーザにとって特別な力を必要としない状態を取ることが望ましいことがわかった。
\subsection{操作方法案2}
操作方法案1の改善点を受け、操作方法案2を考えた。\refImg{sousa2}はそのイメージ図である。

操作方法案2では足を手前に引き、かかとを浮かせた状態をポインタが真ん中に来るものとする。ポインタを下に動かしたいときは、その位置からかかとを下げる。それに伴い、膝も下に下がる。ポインタを上に動かしたいときは操作方法案1と同様にかかとを更に浮かせる。
操作方法案2についても意見を募った。しかし、ポインタを下方向に動かす動作は、足首が本来とは逆の方向に曲がるために、正しい操作が可能かはその人の足首の柔らかさに依存した。したがって、\fixme{足首に負担がかからない動作方法}について、操作方法案を再考することとした。
\subsection{操作方法案3}
操作方法案2の改善点を受け、操作方法案3を考えた。\refImg{sousa3}はそのイメージ図である。
操作方法案3では、かかと含め、足が完全に床についている状態をポインタが画面の真ん中に来るものとする。ポインタを下に動かしたいときは、足を十分引き膝の位置を下げる。ポインタを上に動かしたいときは操作方法案1と同様にかかとを浮かせる。
操作方法案3について意見を募ったところ、操作方法案1で問題になった疲労感、操作方法案2で問題となった足首に対する依存が解消されたとの回答を受けた。したがって、この操作方法でポインタ操作を行うものとする。
\img{htbp}{1}{sousa1.pdf}{操作方法案1}{sousa1}
\img{htbp}{1}{sousa2.pdf}{操作方法案2}{sousa2}
\img{htbp}{1}{sousa3.pdf}{操作方法案3}{sousa3}