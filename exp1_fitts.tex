\chapter{実験:膝によるマウスカーソル操作の性能評価} \chaLabel{exp}
本章では,製作したプロトタイプを用いて膝を用いたマウスカーソル操作の特徴を実験を通して調査する.
\section{目的}
本実験では,膝によるマウスカーソル操作をフィッツの法則\cite{fitts}に当てはめて,その性能を明らかにする.またアンケートから,ユーザの操作性,疲労感を明らかにする.
\section{評価方法}
実験の評価は,フィッツの法則を用いて行う.フィッツの法則は,式\ref{formula:fitts}によって表される.
\begin{eqnarray}
	MT = a + b\log_2{(D/W + 1)}
	\label{formula:fitts}
\end{eqnarray}
式\ref{formula:fitts}に用いられている各係数は以下の通りである.
\begin{itemize}
	\item {$MT$(Moving Time): }ターゲットを選択するまでにかかる時間
	\item {$a,b$: }ユーザと装置に依存する定数
	\item {$D$: }ポインタがある場所からポインティングするターゲットまでの距離(ターゲット間距離)%ここではターゲットが配置されている円の直径に近似する
	\item {$W$: }選択するべきターゲットの幅
	\item { $\log_2{(D/W + 1)}$ [bit]: } 課題の困難度を表す数値\ (Index of Difficulty(ID)と呼ばれる.)
\end{itemize}
IDが高くなればなるほど,ポインティングが難しくなり,MTも大きくなる.
性能の評価にはIDから課題を達成するのに要した時間を割った値(Throuput, TP)が用いられる
%(式\ref{formula:tp})
.
\begin{comment}
	\begin{eqnarray}
	TP = \cfrac{ID}{MT}
	\label{formula:tp}
\end{eqnarray}
\end{comment}

\section{実験手順}
実験には,ISO9241-411\cite{9241411}に記載されている,マルチディレクショナルポインティングタスクに基づいて製作したプログラムを使用した.\refImg{mdpt}は実験に使用したプログラムの動作イメージである.

参加者は円周上に配置された13個のターゲットを,0から13の順に選択する.選択するべきターゲットは水色で示され,それ以外のターゲットは背景と同じ色で表される.ターゲットを1回選択することを1試行と数え,はじめの0番のターゲットの選択を除いた13試行を1タスクと数える.
膝を動かしてカーソルを移動させ,ターゲットとポインタが重なった時に選択操作を行う.本プログラムでは,選択操作は足ではなくキーボード上のEnterキーで行う.
\img{htbp}{1.0}{mdpt.png}{実験に使用したプログラム}{mdpt}

実験条件として,ターゲット幅($W$)とターゲット間距離($D$)を次のように変化させた.

\begin{itemize}
	\item $D$: 2.0,5.0,8.0 (インチ)
	\item $W$: 0.5,1.0,1.5 (インチ)
\end{itemize}
これにより得られる,以下の9つのIDの条件を1タスクずつ行う.これを1セッションと数える.
\begin{itemize}
	\item $ID = $: \{1.22, 1.59, 2.12, 2.32, 2.59, 2.66, 3.17,  3.46, 4.09\}
\end{itemize}
実験は両膝について3セッションずつ行うものとし,片膝について行うことを1ピリオドと数える.したがって,参加者1人につき,2ピリオド(左右の膝)$\times$13(ターゲット数)$\times$9(条件)$\times$3(セッション)=702試行を行う.
1セッション終了ごとに3分間,1ピリオド終了後に10分間の休憩時間を設けた.また,2ピリオド終了後にアンケートを行った.
%アンケートは操作の使いやすさ,快適さ,スムーズさ,肉体的難しさ,精神的難しさと腹部,太もも,ふくらはぎ,足の疲労感を5点リッカート尺度で評価した.
本実験には3名が参加した.全員が男性であり,年齢はそれぞれP1:22歳,P2:24歳,P3:22歳である.P1,P3は左膝・右膝,P2は右膝・左膝の順でそれぞれ実験を行なった.
参加者はセッションの開始前にプロトタイプを設置した机の前に座り,膝を動かすことができる範囲を決定するためのキャリブレーションを行う.したがって,実験全体では6回キャリブレーションを行う.ピリオドの最初のセッションでは,キャリブレーション後に練習時間を5分設けた.

\section{収集データ}
解析のために収集したデータは次のとおりである.
\begin{itemize}
	\item 試行ごとのターゲットの選択に要した時間
	\item その試行でターゲット選択が正しくできたかを表すフラグ
\end{itemize}


\section{実験結果}
\refImg{exp_left}は左膝の,\refImg{exp_right}は右膝の実験結果を表す.横軸は式\ref{formula:fitts}におけるID,縦軸は選択時間であり,グラフには各参加者の選択時間と,選択時間を元に線形回帰で求めた直線が描かれている.
\img{htbp}{1.0}{left.pdf}{左膝の選択時間とそのモデル}{exp_left}
\img{htbp}{1.0}{right.pdf}{右膝の選択時間とそのモデル}{exp_right}
\refImg{exp_error}は両膝のエラー率を表す.横軸は参加者であり,縦軸はエラー率が百分率で表される.グラフには参加者ごとにセッション1,2,3のエラー率が棒グラフで表され,全セッションの平均エラー率が折れ線グラフで表されている.全参加者のエラー率の平均は,左膝で1.14\%,右膝で1.71\%であった.
%今回の実験では,全参加者のエラー率の平均が,左膝で1.14\%,右膝で1.71\%であり,これはVellosoら\cite{velloso:hal-01599657}やHorodniczyら\cite{Horodniczy:2017:FHE:3025453.3025625}が行なった実験よりも低い値を得た.
\img{htbp}{0.9}{error.pdf}{両膝のエラー率}{exp_error}
\img{htbp}{1.0}{tp.pdf}{両膝のスループット}{exp_tp}
\refImg{exp_tp}は参加者ごとに左膝,右膝のスループットを計算した結果である.縦軸はスループットの値で,横軸は3人の参加者と平均を表している.スループットの平均は,左膝で$1.497$[bit/s],右膝で$1.540$[bit/s]であった.t検定を行なったところ,左右の膝のスループットに有意な差はなかった($t=-0.151, df=4$).
