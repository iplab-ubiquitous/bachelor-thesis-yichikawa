\chapter{膝によるマウスカーソル操作の性能評価} \chaLabel{exp}
本章では,製作したプロトタイプを用いて膝を用いたマウスカーソル操作の特徴を実験を通して調査する.
\section{目的}
本実験では,膝によるマウスカーソル操作をフィッツの法則\cite{fitts}に当てはめて,その性能を明らかにする.またアンケートから,ユーザの操作性,疲労感を明らかにする.
\section{評価方法}
実験の評価は,フィッツの法則を用いて行う.フィッツの法則は,式\ref{formula:fitts}によって表される.
\begin{eqnarray}
	MT = a + b\log_2{(D/W + 1)}
	\label{formula:fitts}
\end{eqnarray}
式\ref{formula:fitts}に用いられている各係数は以下の通りである.
\begin{itemize}
	\item {$MT$(Moving Time): }ターゲットを選択するまでにかかる時間
	\item {$a,b$: }ユーザと装置に依存する定数
	\item {$D$: }ポインタがある場所からポインティングするターゲットまでの距離(ターゲット間距離)%ここではターゲットが配置されている円の直径に近似する
	\item {$W$: }選択するべきターゲットの幅
	\item { $\log_2{(D/W + 1)}$ [bit]: } 課題の困難度を表す数値\ (Index of Difficulty(ID)と呼ばれる.)
\end{itemize}
IDが高くなればなるほど,ポインティングが難しくなり,MTも大きくなる.
性能の評価にはIDから課題を達成するのに要した時間を割った値(Throuput, TP)が用いられる
%(式\ref{formula:tp})
.
\begin{comment}
	\begin{eqnarray}
	TP = \cfrac{ID}{MT}
	\label{formula:tp}
\end{eqnarray}
\end{comment}

\section{実験手順}

実験には,ISO9241-411\cite{9241411}に記載されている,マルチディレクショナルポインティングタスクに基づいて製作したプログラムを使用した.\refImg{mdpt}は実験に使用したプログラムの動作イメージである.

参加者は円周上に配置された13個のターゲットを,0から13の順に選択する.選択するべきターゲットは水色で示され,それ以外のターゲットは背景と同じ色で表される.ターゲットを1回選択することを1試行と数え,はじめの0番のターゲットの選択を除いた13試行を1タスクと数える.
膝を動かしてカーソルを移動させ,ターゲットとポインタが重なった時に選択操作を行う.本プログラムでは,選択操作は足ではなくキーボード上のEnterキーで行う.
\img{htbp}{0.7}{mdpt.png}{実験に使用したプログラム}{mdpt}

実験条件として,ターゲット幅($W$)とターゲット間距離($D$)を次のように変化させた.

\begin{itemize}
	\item $D$: 2.0,5.0,8.0 (インチ)
	\item $W$: 0.5,1.0,1.5 (インチ)
\end{itemize}
これにより得られる,以下の9つのIDの条件を1タスクずつ行う.これを1セッションと数える.
\begin{itemize}
	\item $ID = $: \{1.22, 1.59, 2.12, 2.32, 2.59, 2.66, 3.17,  3.46, 4.09\}
\end{itemize}
実験は両膝について3セッションずつ行うものとし,片膝について行うことを1ピリオドと数える.したがって,参加者1人につき,2ピリオド(左右の膝)$\times$13(ターゲット数)$\times$9(条件)$\times$3(セッション)=702試行を行う.
1セッション終了ごとに3分間,1ピリオド終了後に10分間の休憩時間を設けた.また,2ピリオド終了後にアンケートを行った.
%アンケートは操作の使いやすさ,快適さ,スムーズさ,肉体的難しさ,精神的難しさと腹部,太もも,ふくらはぎ,足の疲労感を5点リッカート尺度で評価した.
本実験には3名が参加した.全員が男性であり,年齢はそれぞれP1:22歳,P2:24歳,P3:22歳である.P1,P3は左膝・右膝,P2は右膝・左膝の順でそれぞれ実験を行なった.
参加者はセッションの開始前にプロトタイプを設置した机の前に座り,椅子の高さ,ディスプレイからの距離を参加者の好みに合わせて調節した(机の高さは固定である).椅子の位置は,プロトタイプが設置されている机の位置を表す黒色テープを基準に調整した.その後,膝を動かすことができる範囲を決定するためのキャリブレーションを行う.実験全体では6回キャリブレーションを行う.ピリオドの最初のセッションでは,キャリブレーション後に練習時間を5分設けた.
\section{実験機器}
実験には,プロトタイプ,ディスプレイ(解像度:1920*1080ピクセル,21.5インチ,DELL社製ST2220Lb),ディスプレイを設置する台座,キーボード(Apple社製,Magic Keyboard),
%マウス(Apple社製,Magic Mouse)
を使用した.
%マウスは,キャリブレーションを行うときに実験者のみが使用し,実験時には使用しない.
実験条件の$D$と$W$の数値は,実験プログラムの中で自動的にディスプレイのピクセル密度(102.42[ppi])を元にインチからピクセルに変換している.実験プログラムはディスプレイに映し出され、参加者はディスプレイを見てキーボードとプロトタイプを操作する.\refImg{jikken_image}は,実際の実験の様子である.
\img{htbp}{1.0}{jikken_image.pdf}{実験の様子}{jikken_image}

\section{収集データ}
解析のために収集したデータは次のとおりである.
\begin{itemize}
	\item 試行ごとのターゲットの選択に要した時間
	\item その試行でターゲット選択が正しくできたかを表すフラグ
\end{itemize}


\section{実験結果}
\refImg{exp_left}は左膝の,\refImg{exp_right}は右膝の実験結果を表す.横軸は式\ref{formula:fitts}におけるID,縦軸は選択時間であり,グラフには各参加者の選択時間と,選択時間を元に線形回帰で求めた直線が描かれている.
\refImg{exp_error}は両膝のエラー率を表す.横軸は参加者であり,縦軸はエラー率が百分率で表される.グラフにはセッションごとにP1,P2,P3のエラー率が棒グラフで表され,セッションごとの参加者間の平均エラー率が折れ線グラフで表されている.全参加者のエラー率の平均は,左膝で1.14\%,右膝で1.71\%であった.\refImg{exp_tp}は参加者ごとに左膝,右膝のスループットを計算した結果である.横軸は参加者であり,縦軸はスループットである.グラフにはセッションごとにP1,P2,P3のスループットが棒グラフで表され,セッションごとの参加者間の平均スループットが折れ線グラフで表されている.全体のスループットの平均は,左膝で$1.497$[bit/s],右膝で$1.540$[bit/s]であった.t検定を行なったところ,左右の膝のスループットに有意な差はなかった($t=-0.151, df=4$).
表\ref{tab:anche}にアンケートの結果を示す.操作の使いやすさ,精神的な難しさ,腹部の疲労感,ふくらはぎの疲労感で比較的良好な結果を得た.
%今回の実験では,全参加者のエラー率の平均が,左膝で1.14\%,右膝で1.71\%であり,これはVellosoら\cite{velloso:hal-01599657}やHorodniczyら\cite{Horodniczy:2017:FHE:3025453.3025625}が行なった実験よりも低い値を得た.
\img{htbp}{0.9}{left.pdf}{左膝の選択時間とそのモデル}{exp_left}
\img{htbp}{0.9}{right.pdf}{右膝の選択時間とそのモデル}{exp_right}
\img{htbp}{0.9}{error.pdf}{両膝のエラー率}{exp_error}
\img{htbp}{0.9}{tp.pdf}{両膝のスループット}{exp_tp}
\begin{table}[htbp]
	\begin{center}
		\caption{アンケートの結果}
		\begin{tabular}{|c|c|c|c|c|}
		\hline 
	   		& P1 & P2 & P3 & 平均   \\ \hline
			\begin{tabular}{c}操作の使いやすさ\\ (1: 使いにくい - 5:使いやすい)\end{tabular}  & 4  & 4  & 3  & 3.67 \\ \hline
			\begin{tabular}{c}操作の快適さ\\ (1: 快適でない- 5:快適である)\end{tabular}  & 4  & 3  & 3  & 3.33 \\ \hline
			\begin{tabular}{c}操作のスムーズさ\\ (1: スムーズでない - 5:スムーズである)\end{tabular}  & 4  & 3  & 3  & 3.33 \\ \hline
			\begin{tabular}{c}肉体的な難しさ\\ (1: 簡単である - 5:難しい)\end{tabular}  & 3  & 1  & 4  & 2.67 \\ \hline
			\begin{tabular}{c}精神的な難しさ\\ (1: 簡単である - 5:難しい)\end{tabular}  & 2  & 1  & 2  & 1.67 \\ \hline
			\begin{tabular}{c}腹部の疲労感\\(1: 疲れていない - 5:疲れている)\end{tabular}& 1  & 1  & 1  & 1.00 \\ \hline
			\begin{tabular}{c}太ももの疲労感\\(1: 疲れていない - 5:疲れている)\end{tabular}& 4  & 1  & 3  & 2.67 \\ \hline
			\begin{tabular}{c}ふくらはぎの疲労感\\(1: 疲れていない - 5:疲れている)\end{tabular}& 3  & 1  & 2  & 2.00 \\ \hline
			\begin{tabular}{c}足の疲労感\\(1: 疲れていない - 5:疲れている)\end{tabular}& 4 & 1  & 3  & 2.67 \\ \hline
		
		\end{tabular}
		\label{tab:anche}
	\end{center}
\end{table}

\section{考察}
\subsection{エラー率とキャリブレーション}
\refCha{exp}で行なった実験の結果,膝による操作での選択ミス率は左膝で1.14\%,右膝で1.71\%であり,Horodniczyら\cite{Horodniczy:2017:FHE:3025453.3025625}やVellosoら\cite{velloso:hal-01599657}の手法の結果よりも低い値であった.原因として以下の3点が考えられる.

1点目はキャリブレーションを1セッションごと,計6回と多く行なっていたことである.このため,キャリブレーションの回数を減らすことでエラー率やスループットにどのような影響を与えるかを調査する必要がある.
2点目は膝はカーソル操作のみを行い,選択操作はキーボードのEnterキーで行なっていたことである.したがって,マウスの機能をすべて膝と足のみを使って実現した場合,エラー率が変化する可能性がある.
3点目はポインティングタスクの難易度である.今回の実験では,IDの最大は4.09[bit]であったが,最大5.67[bit]で行う実験も存在した\cite{Horodniczy:2017:FHE:3025453.3025625, velloso:hal-01599657}.したがって,よりIDの高い条件での実験を行う必要がある.

\subsection{操作性と疲労感の改善}

\refCha{kneeMoving}にて,疲労感を考慮して膝の移動方法を設計した.しかし太ももと足の疲労は比較的疲労感が高い.今後膝の移動方法をさらに見直し,疲労感を改善する工夫が必要である.さらに,操作の快適さ,スムーズさにも改善の余地があるとわかる.ハードウェア設計の改善を行うことで,操作性についても改善を図る.




