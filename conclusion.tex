\chapter{結論}\chaLabel{conclusion}
本研究では特別な装置を足に装着することなく,かつ簡単な設置方法で足によるコンピュータの操作を可能にすることを目的とし,アプローチとして机の裏に設置した距離センサアレイとArduinoからなるハードウェアと,膝の位置を認識するソフトウェアからなるプロトタイプを製作し,膝の位置を認識するためのプログラムを実装した.また,膝の位置をコンピュータ上のマウスカーソルの位置に反映させ,膝を用いたマウスカーソル操作に適用した.特徴や性能を調査するために,ISO9241-411に準拠したマルチディレクショナルポインティングタスクを用いた実験を行い,フィッツの法則モデルを示した.今後は膝の運動の調査とそれに合わせたハードウェアの設計,条件やキャリブレーションの回数を変化させた実験などを行い,さらなる操作性の向上を目指す.