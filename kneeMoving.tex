\chapter{ユーザの膝の動作}\chaLabel{kneeMoving}
\section{概要}
マウスやタッチパッドの操作と異なり,膝は前方や後方に動かすことはできない.そのため,本研究で想定する膝の移動は,膝を傾けることによる左右方向と,かかとを浮かせたり床につけたりすることによる上下方向の移動である.\refImg{ov}は膝でマウスカーソルを操作するときのイメージである.ユーザは水色の矢印で示される方向に膝を動かし,システムは鉛直面における膝の2次元座標を計算する.

\img{htbp}{1}{sousa_overview.pdf}{膝によるマウスカーソル操作のイメージ}{ov}
\section{膝の移動方法}
\subsection{左右方向}
ユーザはマウスカーソルを左右に移動させたい時には,膝を左右に移動する.\refImg{sousalr1},\refImg{sousalr2}は足を正面から見たときの左右方向の移動のイメージである.\refImg{sousalr1}のように,このとき足先も同時に移動させ,足全体を移動させると,太ももに疲労が生じる.また,足に装置を取り付け,摩擦力を小さくするということも目的に反する.そこで\refImg{sousalr2}のように,左右方向に移動する時に限っては,足の位置をなるべく固定し,膝を左右に傾けることで移動することとした.
\img{H}{1.0}{lr1.pdf}{左右方向の操作イメージ1}{sousalr1}
\img{H}{1.0}{lr2.pdf}{左右方向の操作イメージ2}{sousalr2}
%\subsection{操作方法案1}
%マウスの操作では,カーソルを$x$軸方向の操作にはマウスを左右に動かす方法,$y$軸方向の操作にはマウスを\fixme{奥に押す},または手前に引くという動作を行う.これらを膝に置き換えた時,
\subsection{上下方向}
上方向に移動させたい時は,かかとを浮かせて膝を机に近づける.逆に下方向に移動させたい時は,足を手前に引き,その時に浮いたかかとを床に近づけることで,膝を机から遠ざける.\refImg{sousa1},\refImg{sousa3}は,足を横から見たときの上下方向の移動のイメージである.
\img{htbp}{1}{sousa1.pdf}{上下方向の操作イメージ1}{sousa1}
\img{htbp}{1}{sousa3.pdf}{上下方向の操作イメージ2}{sousa3}

\refImg{sousa1}に示されているような移動方法は,足が完全に地面についている時に,マウスカーソルは画面の一番下の位置になってしまう.そのため,ユーザが画面の真ん中付近にカーソルを移動する時にかかとを浮かせた状態を維持しなければならず,疲労が生じてしまう.そこで\refImg{sousa3}のように移動することで,足が完全に地面についている時にカーソルが真ん中にあるため,ユーザは比較的楽な姿勢で操作することができる.


%適用することができるが,膝を奥に動かしたり手前に引くということはできない.したがって,$y$軸方向の操作として,膝を持ち上げる動作を適用することとした.\refImg{sousa1}は$y$軸方向の操作方法として考案した膝の動かし方の1つである.画面の下の方へカーソルを動かす時には,かかとを下げ,カーソルが下限にくる時,かかとは完全に床につく状態になる.逆に画面の上の方へカーソルを動かす時には,かかとを浮かせる.


%しかし,この操作方法について意見を募ったところ,かかとを浮かせた状態で維持することが困難であり,著しく疲労を感じるという回答を受けた.特に,カーソルを画面の真ん中で維持することが困難であった.したがって,カーソルを画面の上下方向の真ん中に維持する時,膝の姿勢がユーザにとって特別な力を必要としない状態を取ることが望ましいことがわかった.
%\subsection{操作方法案2}
%操作方法案1の改善点を受け,操作方法案2を考えた.\refImg{sousa2}はそのイメージ図である.
%操作方法案2では足を手前に引き,かかとを浮かせた状態をカーソルが真ん中に来るものとする.カーソルを下に動かしたいときは,その位置からかかとを下げる.それに伴い,膝も下に下がる.カーソルを上に動かしたいときは操作方法案1と同様にかかとを更に浮かせる.
%操作方法案2についても意見を募った.しかし,カーソルを下方向に動かす動作は,足首が本来とは逆の方向に曲がるために,正しい操作が可能かはその人の足首の柔らかさに依存した.したがって,\fixme{足首に負担がかからない動作方法}について,操作方法案を再考することとした.
%\subsection{操作方法案3}
%操作方法案2の改善点を受け,操作方法案3を考えた.\refImg{sousa3}はそのイメージ図である.
%操作方法案3では,かかと含め,足が完全に床についている状態をカーソルが画面の真ん中に来るものとする.カーソルを下に動かしたいときは,足を十分引き膝の位置を下げる.カーソルを上に動かしたいときは操作方法案1と同様にかかとを浮かせる.
%操作方法案3について意見を募ったところ,操作方法案1で問題になった疲労感,操作方法案2で問題となった足首に対する依存が解消されたとの回答を受けた.したがって,この操作方法でカーソル操作を行うものとする.