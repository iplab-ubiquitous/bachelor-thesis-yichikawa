\chapter{膝によるマウスカーソル操作手法}\chaLabel{kneeMoving}
\section{概要}
本研究では膝の位置を認識し,
%\refSubsec{word}で述べたような
マウスカーソル操作への応用を行う.
%これは,マウスカーソルのような繊細な操作を膝で行ったときの操作性や疲労感を明らかにし,膝による操作全体の改良を図るためである. 
マウスやタッチパッドの操作と異なり,膝は前方や後方に動かすことはできない.そのため,本研究で想定する膝の移動は,膝を傾けることによる左右方向と,かかとを浮かせたり床につけたりすることによる上下方向の移動である.\refImg{ov}は膝でマウスカーソルを操作するときのイメージである.ユーザは水色の矢印で示される方向に膝を動かし,システムは鉛直面における膝の2次元座標を計算する.

\img{htbp}{1}{sousa_overview.pdf}{膝によるマウスカーソル操作のイメージ}{ov}
\section{膝の移動方法}
\subsection{左右方向}
ユーザはマウスカーソルを左右に移動する時には,膝を左右に傾ける.\refImg{sousalr2},足を正面から見た,膝を左右に傾けるイメージである.\refImg{sousalr1}のように,足先も同時に移動させ,足全体を移動させると,足と床面との間に摩擦が生じて、移動が難しくなる上に,太ももにも疲労が生じる.\refImg{sousalr2}のように,足を傾け足の位置を固定することで,操作を簡単化し,太ももへの疲労を削減する.
\img{H}{1.0}{lr2.pdf}{膝を左右に傾けるイメージ}{sousalr2}
\img{H}{1.0}{lr1.pdf}{足全体を移動させるイメージ}{sousalr1}

%\subsection{操作方法案1}
%マウスの操作では,カーソルを$x$軸方向の操作にはマウスを左右に動かす方法,$y$軸方向の操作にはマウスを\fixme{奥に押す},または手前に引くという動作を行う.これらを膝に置き換えた時,
\subsection{上下方向}
上方向に移動する時は,かかとを浮かせて膝を机に近づける.逆に下方向に移動する時は,足を手前に引き,その時に浮いたかかとを床に近づけることで,膝を机から遠ざける.\refImg{sousa3},\refImg{sousa1}は,足を横から見たときの上下方向の移動のイメージである.
\img{htbp}{1}{sousa3.pdf}{かかとを浮かせて膝を近づけ,足を引き膝を遠ざけるイメージ}{sousa3}
\img{htbp}{1}{sousa1.pdf}{かかとを浮かせるだけで,膝を近づけたり遠ざけたりするイメージ}{sousa1}
\refImg{sousa1}に示されているような移動方法は,足が完全に地面についている時に,マウスカーソルは画面の一番下の位置になってしまう.そのため,ユーザが画面の真ん中付近にカーソルを移動する時にかかとを浮かせた状態を維持しなければならず,疲労が生じてしまう.そこで\refImg{sousa3}のように移動することで,足が完全に地面についている時にカーソルが真ん中にあるため,ユーザは比較的楽な姿勢で操作することができる.

\section{利用イメージ}
\subsection{ワープロソフト利用時のマウスカーソル操作}
\subsecLabel{word}
\refImg{usc1}にイメージ図を示す.ワープロソフトを利用する時,手はキーボード上にあることが多い.しかし,文字色や文字サイズの変更,図の挿入など,コマンド入力で行うことができない操作を行う時,ユーザはマウスやタッチパッドに手を移動しなければならない.本手法を用いて,膝によってマウスカーソルを操作し,フットスイッチなどでクリック操作を行うことで,ユーザは手の移動を削減することができる.
\img{htbp}{1.0}{usc1.pdf}{ワープロソフトの利用時に膝でマウスカーソルを操作するイメージ}{usc1}
\subsection{画面に表示した回路図を見ながら電子工作をするときの画面操作}
\refImg{usc2}にイメージ図を示す.ユーザは画面では回路図を写しながら,その回路図に従って電子回路を製作する.ユーザは回路図をズームするとき,作業を一時中断して,マウスやタッチパッドを手で操作しなくてはならない.本手法を用いて,膝でマウスカーソルを操作し,足を用いてペダルを操作しズームを行う.これによりユーザは作業に集中したままズーム操作を行うことができる.
\img{htbp}{1.0}{usc2.pdf}{画面を見ながら作業をするイメージ}{usc2}





%適用することができるが,膝を奥に動かしたり手前に引くということはできない.したがって,$y$軸方向の操作として,膝を持ち上げる動作を適用することとした.\refImg{sousa1}は$y$軸方向の操作方法として考案した膝の動かし方の1つである.画面の下の方へカーソルを動かす時には,かかとを下げ,カーソルが下限にくる時,かかとは完全に床につく状態になる.逆に画面の上の方へカーソルを動かす時には,かかとを浮かせる.


%しかし,この操作方法について意見を募ったところ,かかとを浮かせた状態で維持することが困難であり,著しく疲労を感じるという回答を受けた.特に,カーソルを画面の真ん中で維持することが困難であった.したがって,カーソルを画面の上下方向の真ん中に維持する時,膝の姿勢がユーザにとって特別な力を必要としない状態を取ることが望ましいことがわかった.
%\subsection{操作方法案2}
%操作方法案1の改善点を受け,操作方法案2を考えた.\refImg{sousa2}はそのイメージ図である.
%操作方法案2では足を手前に引き,かかとを浮かせた状態をカーソルが真ん中に来るものとする.カーソルを下に動かしたいときは,その位置からかかとを下げる.それに伴い,膝も下に下がる.カーソルを上に動かしたいときは操作方法案1と同様にかかとを更に浮かせる.
%操作方法案2についても意見を募った.しかし,カーソルを下方向に動かす動作は,足首が本来とは逆の方向に曲がるために,正しい操作が可能かはその人の足首の柔らかさに依存した.したがって,\fixme{足首に負担がかからない動作方法}について,操作方法案を再考することとした.
%\subsection{操作方法案3}
%操作方法案2の改善点を受け,操作方法案3を考えた.\refImg{sousa3}はそのイメージ図である.
%操作方法案3では,かかと含め,足が完全に床についている状態をカーソルが画面の真ん中に来るものとする.カーソルを下に動かしたいときは,足を十分引き膝の位置を下げる.カーソルを上に動かしたいときは操作方法案1と同様にかかとを浮かせる.
%操作方法案3について意見を募ったところ,操作方法案1で問題になった疲労感,操作方法案2で問題となった足首に対する依存が解消されたとの回答を受けた.したがって,この操作方法でカーソル操作を行うものとする.