% !TEX root = ../interaction2017_shima.tex

% 手法名
\newcommand{\SysName}{本プロトタイプ}
\newcommand{\selection}{選択ジェスチャ}
\newcommand{\operation}{操作ジェスチャ}

% 修正とTODO
\newcommand{\fixme}[1]{\textcolor[rgb]{1,0,0}{#1}}
\newcommand{\TODO}[1]{\textcolor[rgb]{0,0,1}{#1}}
%本番は↓のコメントアウトを外すこと!!!
%\renewcommand{\TODO}[1]{} \renewcommand{\fixme}[1]{}

% 参照
\newcommand{\refImg}[1]{図\ref{img:#1}}
\newcommand{\refSec}[1]{\ref{sec:#1}節}
\newcommand{\refSubsec}[1]{\ref{ssec:#1}項}
\newcommand{\refCha}[1]{第\ref{cha:#1}章}
\newcommand{\refEq}[1]{式\ref{eq:#1}}
%\newcommand{\refSec}[1]{第\ref{sec:#1}節}
\newcommand{\secLabel}[1]{\label{sec:#1}}
\newcommand{\subsecLabel}[1]{\label{ssec:#1}}
\newcommand{\chaLabel}[1]{\label{cha:#1}}

% 画像
\newcommand{\img}[5]{
\begin{figure}[#1]
	\begin{center}
		\includegraphics[width = #2\hsize]{./img/#3}
	\end{center}
	\caption{#4}
	\label{img:#5}
\end{figure}
}

% 複数コラムある場合のぶち抜き画像
\newcommand{\IMG}[5]{
\begin{figure*}[#1]
	\begin{center}
		\includegraphics[width = #2\hsize]{./img/#3}
	\end{center}
	\caption{#4}
	\label{img:#5}
\end{figure*}
}

\newcommand{\unit}[1]{\,#1} % 120,\unit{mm}→120mm」をイイ!感じにスペーシングしてくれる
\newcommand{\kake}{~$\times$~} % →×

% 情報処理学会のbibtexスタイル(ipsjunsrt.bstなど)を使うときにこれがないと,\newblockが定義されてないよってエラーになる
% 参考:http://gaso.hatenablog.com/entry/2014/05/11/%E6%9F%90%E5%AD%A6%E4%BC%9A%E3%83%86%E3%83%B3%E3%83%97%E3%83%AC%E3%81%AE%E3%82%B9%E3%82%BF%E3%82%A4%E3%83%AB%E3%83%95%E3%82%A1%E3%82%A4%E3%83%AB%E3%81%A7BibTeX%E3%82%92%E4%BD%BF%E3%81%A3%E3%81%9F%E6%99%82
\def\newblock{\hskip .11em plus .33em minus .07em}


