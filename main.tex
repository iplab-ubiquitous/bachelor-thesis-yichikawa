%%
% このファイルは,筑波大学情報学群情報科学類の
% 卒業研究論文本体のサンプルです.
% このファイルを書き換えて,この例と同じような書式の論文本体を
% LaTeXを使って作成することができます.
%
% コンピュータ環境や,LaTeX環境の設定によっては漢字コードや改行コードを
% 変更する必要があります.
%%
\documentclass[a4paper,11pt]{jreport}

%%【PostScript, JPEG, PNG等の画像の貼り込み】
%% 利用するパッケージを選んでコメントアウトしてください.
%\usepackage{graphicx} % for \includegraphics[width=3cm]{sample.eps}
\usepackage[dvipdfmx]{graphicx, color}
%\usepackage{epsfig} % for \psfig{file=sample.eps,width=3cm}
%\usepackage{epsf} % for \epsfile{file=sample.eps,scale=0.6}
%\usepackage{epsbox} % for \epsfile{file=sample.eps,scale=0.6}

%% dvipdfm を使う場合(dvi->pdfを直接生成する場合)
%\usepackage[dvipdfmx]{color}
%% dvipdfm を使ってPDFの「しおり」を付ける場合
%\usepackage[dvipdfm,bookmarks=true,bookmarksnumbered=true,bookmarkstype=toc]{hyperref}
%% 参考:dvipdfm 日本語版
%% http://hamilcar.phys.kyushu-u.ac.jp/~hirata/dvipdfm/
\usepackage[left=25truemm,top=35truemm,right=25truemm,bottom=50truemm]{geometry}
\usepackage{times} % use Times Font instead of Computer Modern
\usepackage{comment}
\usepackage{here}

\setcounter{tocdepth}{3}
\setcounter{page}{-1}

\setlength{\parskip}{0em}
\setlength{\topsep}{0em}

%\newcommand{\zu}[1]{{\gt \bf 図\ref{#1}}}

%% タイトル生成用パッケージ(重要)
\usepackage{coins-jp-utf8}

\usepackage{siunitx}
\usepackage{amsmath}
\usepackage[margin=5pt]{subcaption}
% !TEX root = ../interaction2017_shima.tex

% 手法名
\newcommand{\SysName}{本プロトタイプ}
\newcommand{\selection}{選択ジェスチャ}
\newcommand{\operation}{操作ジェスチャ}

% 修正とTODO
\newcommand{\fixme}[1]{\textcolor[rgb]{1,0,0}{#1}}
\newcommand{\TODO}[1]{\textcolor[rgb]{0,0,1}{#1}}
%本番は↓のコメントアウトを外すこと!!!
%\renewcommand{\TODO}[1]{} \renewcommand{\fixme}[1]{}

% 参照
\newcommand{\refImg}[1]{図\ref{img:#1}}
\newcommand{\refSec}[1]{\ref{sec:#1}節}
\newcommand{\refSubsec}[1]{\ref{ssec:#1}項}
\newcommand{\refCha}[1]{第\ref{cha:#1}章}
\newcommand{\refEq}[1]{式\ref{eq:#1}}
%\newcommand{\refSec}[1]{第\ref{sec:#1}節}
\newcommand{\secLabel}[1]{\label{sec:#1}}
\newcommand{\subsecLabel}[1]{\label{ssec:#1}}
\newcommand{\chaLabel}[1]{\label{cha:#1}}

% 画像
\newcommand{\img}[5]{
\begin{figure}[#1]
	\begin{center}
		\includegraphics[width = #2\hsize]{./img/#3}
	\end{center}
	\caption{#4}
	\label{img:#5}
\end{figure}
}

% 複数コラムある場合のぶち抜き画像
\newcommand{\IMG}[5]{
\begin{figure*}[#1]
	\begin{center}
		\includegraphics[width = #2\hsize]{./img/#3}
	\end{center}
	\caption{#4}
	\label{img:#5}
\end{figure*}
}

\newcommand{\unit}[1]{\,#1} % 120,\unit{mm}→120mm」をイイ!感じにスペーシングしてくれる
\newcommand{\kake}{~$\times$~} % →×

% 情報処理学会のbibtexスタイル(ipsjunsrt.bstなど)を使うときにこれがないと,\newblockが定義されてないよってエラーになる
% 参考:http://gaso.hatenablog.com/entry/2014/05/11/%E6%9F%90%E5%AD%A6%E4%BC%9A%E3%83%86%E3%83%B3%E3%83%97%E3%83%AC%E3%81%AE%E3%82%B9%E3%82%BF%E3%82%A4%E3%83%AB%E3%83%95%E3%82%A1%E3%82%A4%E3%83%AB%E3%81%A7BibTeX%E3%82%92%E4%BD%BF%E3%81%A3%E3%81%9F%E6%99%82
\def\newblock{\hskip .11em plus .33em minus .07em}




%%
\newcommand{\iic}{\text{I}$^\text{2}$\text{C}}


%% タイトル
%% 【注意】タイトルの最後に\\ を入れるとエラーになります
%\title{密閉容器における受動音響センシングによる内容量推定}
\multilinetitle{机の裏に設置した距離センサアレイによる\\膝位置認識とカーソル操作への応用}
%% 著者
\author{市川 佑}
%% 指導教員
\advisor{高橋 伸\ \ 志築 文太郎}

%% 専攻名 と 年月 (提出年月)
%% 年月は必要に応じて書き替えてください.
\heiseiyear{30}  % 平成の年度
%\majorfield{ソフトウェアサイエンス主専攻}
\majorfield{情報システム主専攻}
%\majorfield{知能情報メディア主専攻}

\begin{document}
\maketitle
\thispagestyle{empty}
\newpage

\thispagestyle{empty}
\vspace*{20pt plus 1fil}
\parindent=1zw
\noindent
%%
%% 論文の概要(Abstract)
%%
\begin{center}
{\Large \bf 要  旨}
\vspace{2cm}
\end{center}
机上でコンピュータを使う作業中に,両手はキーボードの操作を行ったまま,足でマウスカーソルを操作する研究が行われている.しかし,先行研究では足に装置を取りつけなければならない,足先を用いた操作が多く,他の足を使ったインタラクションと組み合わせることができないといった問題点が存在する.本研究では,机裏に設置した距離センサアレイにより,膝の位置を認識し,マウスカーソル操作に応用することを提案する.距離センサ,Arduino,コンピュータからなるハードウェアと,距離データのフィルタ処理,膝の位置認識,キャリブレーション,膝の位置からマウスカーソル座標への変換を行うソフトウェアからなるプロトタイプを実装した.また,プロトタイプを用いて実験を行い,膝によるマウスカーソル操作の性能を評価した.実験結果からプロトタイプの設計や,ユーザの操作性と疲労度に関する今後の課題を示す.

%%%%%
\par
\vspace{0pt plus 1fil}
\newpage

\pagenumbering{roman} % I, II, III, IV
\tableofcontents
\listoffigures
\listoftables

\pagebreak \setcounter{page}{1}
\pagenumbering{arabic} % 1,2,3

\chapter{序論}\chaLabel{background}

\section{背景}
自動車のアクセルやブレーキペダル,ピアノやオルガンのペダルに代表されるように,我々は日常的に足による操作を行なっている.しかし,パーソナルコンピュータをデスクトップ上で作業をする際,我々は手を中心に操作を行い,足に操作が割り当てられることはない.足による操作を用いたコンピュータ向けインタフェースの研究は,1960年代から存在している\cite{1698228}が,現在は手による操作が中心である.しかし,Multitoe\cite{Augsten:2010:MHI:1866029.1866064}のようなタッチ認識を可能にした床面とのインタラクションや,手がふさがった状態におけるモバイル機器の操作\cite{Fan:2017:ESF:3123021.3123043, okumura_2011}など,新たに足によるインタラクションの研究は関心が高まっている.\par
その中でも,デスクトップでの特にコンピュータを使った作業中は両手をキーボードの操作に充てることが多い.そのため,足によるマウスのようなカーソルやポインタの操作を行う研究が盛んである.
%足によってマウスのような操作を行うことで,両手は他の操作を行ったまま,マウスポインタの制御が可能になる.
そのアプローチは,装置を足で動かす方法\cite{Pearson:1986:MMD:22627.22392, Pearson:1988:EET:49108.1046356},可変摩擦機構を取り付けた靴\cite{Horodniczy:2017:FHE:3025453.3025625}があるが,これらは体の一部に装置を取り付けるあるいは大型な装置を用いるものであるため,衣服などに制限が生じたり持ち運びができないなどの制約が加わってしまう.
%このような物理的制約を不要にする研究は少ない\cite{velloso:hal-01599657}.



\section{目的・アプローチ}
前節で述べた問題を解決するために,本研究の目的は,特別な装置を足に装着することなく,かつ簡単に,足を用いたコンピュータの操作を可能にすることである.そのアプローチとして,机下に取り付けた装置から膝の動作を読み取る.膝は足と比べて動かす時に動作が大きくなりにくく,足元よりも動かせる範囲が大きいと考える.また,先行研究では足先を用いる操作が多いのに対し,膝を使うものは少ない\cite{1698228}ため,膝と足による操作の組み合わせによりさらなるインタラクションの拡張が可能である.

本研究では膝の位置を認識し,マウスカーソルの操作に適用する.膝の位置の認識には市販の距離センサを用い,これを10個並べた距離センサアレイと,Arduino,ソフトウェアからなるプロトタイプを製作する.ユーザは距離センサアレイを机の裏に設置し,机の下で膝を上下左右に動かすことでマウスカーソルの操作を行うことができる.
%膝の動作を読み取るシステムの開発を行い,膝を使ってポインタ操作を行う手法を提案する.ユーザは机の下で膝を動かすことで,パーソナルコンピュータ上のポインタを操作することができる.システムのハードウェアには,距離センサを用いた.
%距離センサを用いたトラッキング技術には,例えばAIRBAR\cite{AIRBAR}やLumiwatch\cite{Xiao:2018:LOP:3173574.3173669}が挙げられるが,膝に適用した例はない.また,
%カメラを用いないことで,設置や構築が容易であるという利点があるために,カフェのテーブルや,ホテルのデスクのように様々な場所で設置,利用可能である.\\

\section{貢献}
本研究の貢献を以下に述べる.
\begin{itemize}
	\item 特別な装置を装着することがない膝の位置認識の提案と,膝によるマウスポインタ操作への応用を示した.
	\item 距離センサ10個を用いたプロトタイプを製作し,膝の位置を認識し,マウスカーソル操作へ応用した.
	\item プロトタイプを用いて,フィッツの法則に基づく実験を行なった.
\end{itemize}
\section{本論文の構成}
\refCha{background}では,本研究の背景,目的とそれに対するアプローチについて述べた.\refCha{relatedwork}では,本研究に関連する研究について述べる.\refCha{system}では,本研究で提案する膝の認識におけるユーザの膝の動かし方について述べる.\refCha{implementation}では,膝の位置を認識し,マウスカーソルの座標に反映するためのプロトタイプの実装について述べる.\refCha{exp}では,プロトタイプを用いて行なった,フィッツの法則に基づいた実験とその結果について述べる.\refCha{discussion}では,プロトタイプについて現在わかっている問題点とそれに対する改善の方針を述べる.\refCha{conclusion}では,本研究の結論を述べる.

\begin{comment}
	\fixme{
	\begin{itemize}
		\item 本研究では膝によるマウスカーソル操作を調査する?
		\item 足を使ってみたい
		\item 足先の研究しかない,膝と組み合わせることで様々なインタラクションが可能になる
		\item 膝使ったものは少なく問題がある
		\item 膝を使ったことの理由→足の既存手法と組み合わせることができる,膝の可動域が広い(が先行研究が少ない
		\item いずれにせよ,関連研究が終わるまでに「膝によるマウスカーソル操作」という話に落とし込む
	\end{itemize}
}
\end{comment}

\chapter{関連研究}\chaLabel{relatedwork}
本章では,関連する研究について述べる.まず,現在広く行われている,足によるジェスチャ入力を行う研究について示す.次に本研究で想定する,机上のコンピュータの操作を想定した足を用いた入力の研究を示す.最後に,膝による入力を用いた研究を示す.
%1
\section{足をジェスチャ入力として用いる研究}
足を用いたコンピュータへの入力の目的はいくつか存在する.まず,屋外でスマートフォンなどを操作する時に荷物を持っている,手が汚れているというすぐに手を用いることができない状況を想定したものがある.
Alexanderら\cite{Alexander:2012:PYB:2207676.2208575}は,モバイル端末で頻繁に用いられる操作に対し,足ジェスチャを割り当てるための調査を行なった.	
Fanら\cite{Fan:2017:ESF:3123021.3123043}は,
%荷物を持っている,手が汚れているといった状態において,
足のジェスチャによりモバイル端末を操作することに対する実証研究を行なった.ユーザ定義の足のジェスチャを用いた方法と,荷物を降ろして手で端末を持ち操作する方法を比較したところ,前者の方が70\%高速な操作が可能であるという結果となった.
HanらのKick\cite{Han:2011:KIU:2037373.2037379}では,蹴り出すジェスチャを端末操作に用いるために,ユーザがキックの方向と速度をどの程度制御できるかを調査した.
奥村\cite{okumura_2011}は,靴に加速度と角速度を取得することができるセンサを取り付け,外出時におけるモバイル端末の操作を行うシステムを開発した.
本研究は,屋内での利用のみを想定した環境設置型の装置を用いるという点と,ラップトップコンピュータやデスクトップコンピュータへの利用を想定しているという点から,インタラクションの目的は異なる.

次に屋内環境に設置する装置を用いた研究事例を紹介する.
AugstenらによるMultitoe\cite{Augsten:2010:MHI:1866029.1866064}では,巨大なタッチパネルを床面に設置し,複数の足の認識や足の重心位置の認識した.これにより,床面に表示されたメニューやキーボードを足で操作することを提案した.
鈴木\cite{ssuzuki_2009}は,測域センサによって足の動きをセンシングし,床面におけるインタラクション手法を提案している.
これらの調査やインタラクション技術は立った状態を想定しているものである.本研究では,デスクトップ上での作業中という限定された環境におけるインタラクション手法の提案を目指す.

%2
\section{机上のコンピュータの操作を想定した足を用いた入力の研究}

本節では,机上におけるコンピュータの操作に足を用いた入力手法を調査,提案した研究を紹介する.
Felberbaumら\cite{Felberbaum:2018:BUF:3173574.3173908}は,立った状態,座った状態,投影された画面の上にいる状態の3条件で,GUIに関する操作,仮想空間に関する操作の2種類に対して,どの足ジェスチャを用いるのが好ましいかを,ユーザに対する調査で明らかにした.
Saundersら\cite{Saunders:2016:TFI:2901790.2901815}は,立った状態でのデスクトップアプリケーションの制御に足を用いた.
Pearsonら\cite{Pearson:1986:MMD:22627.22392, Pearson:1988:EET:49108.1046356}は「モル」という装置を開発し,ポインタの操作などに手の代わりに足を使用する方法を調査した.モルを用いた場合でも,訓練によって小さなターゲットを選択することが可能になることを示した.
Horodniczyら\cite{Horodniczy:2017:FHE:3025453.3025625}は,ユーザの靴に可変摩擦式の装置を取り付け,足によるカーソル操作の補助装置として用いた.靴底には低摩擦材と高摩擦材の2つを取り付け,高摩擦材の接地圧力をステッピングモータで制御する.足の位置をカメラにより取得し,ターゲットに近づくにつれ圧力を高める.
Vellosoら\cite{velloso:hal-01599657}は,座っている状態の机の下の足の動きの特徴を調査した.机の下に配置したカメラから,片方の足のつま先をマウス操作に割り当て,1次元と2次元におけるポインティング作業により,パフォーマンスのテストを行なっている.
田中ら\cite{110004704997}は,足の指をマウス操作に用いるために,母指の力制御と運動特性を調査した.
これらの研究では,大型の装置を用いているために持ち運びや設置が困難であったり,靴に対して装置を取り付けるためにユーザに身体上の制約を強いてしまう.
%また,これらの研究で用いられているのは足先であり,
本研究では小型で設置が簡単かつ足に装置を取り付けないアプローチで,問題の解決を図る.また,本研究は足先でなく膝に焦点を当てることで,既存手法との組み合わせによってさらなるインタラクションの拡張を図ることも可能である.

足と他の入力モダリティとの組み合わせを行なった研究という点では,次のようなものが存在する.
G\"{o}belら\cite{Gobel:2013:GFI:2468356.2479610}が提案する手法では視線と足を組み合わせ,視線位置におけるパンとズームの操作を足によるペダル操作で行うことを提案した.
Rajanna\cite{Rajanna:2016:GFI:2876456.2876462}は,視線によるポインティングと足によるクリックコマンドで構成されるシステムを構築した.
本研究では膝による入力操作を行うことで,足を用いた他の手法との組み合わせの可能性を探る.

\section{膝を入力手法として用いる研究}
\img{htbp}{0.7}{eng.png}{English\cite{1698228}らの膝操作手法}{engknee}
膝に関する研究の中で,コンピュータへの入力を想定したものは少ない.
Englishら\cite{1698228}は,テキスト選択において,膝を含めたいくつかの装置やデバイスを用いた時の操作時間を調査した.調査の結果,膝による操作は最も短い時間で選択することができることがわかった.
この論文では,机の下に取り付けた装置のレバーを膝で動かすことで入力を行なったが,この装置は調査を行うために作られた原始的なものであった(\refImg{engknee}).本研究では,距離センサを使うことで,設計の工夫を行なった.
%我々は単純な構造のプロトタイプを開発することで,この研究に貢献する.

%	\begin{itemize}
%		\item 足ジェスチャの研究が様々ある→机の下という環境を想定したものが少ない
%		\item 机の下という環境を想定した研究が少数だが存在する→先述の問題点がある,膝を用いれば組み合わせることができる
%		\item 膝を用いた研究はほぼない→装置が大型である,膝の研究が少ない
%	\end{itemize}




%\chapter{膝位置認識システム}\chaLabel{system}
\section{概要}
本研究では,机の下の片方の膝の位置を認識するシステムを提案する.システムの流れは以下のようになっている.
\begin{enumerate}
	\item ユーザは机の裏に距離センサアレイを設置し,自分のPCと接続する.その後机の前に座り,膝を上下左右に動かす.
	\item 距離センサアレイが,それぞれセンサと膝との距離を計測する.
	\item 全ての距離センサの計測値は膝位置計算ソフトウェアに送信される.
	\item ソフトウェアは,机に垂直な面において膝がどの位置にあるかを計算する.
	\item 膝の位置に応じてアプリケーションを実行する.
\end{enumerate}
この時ユーザは机の裏に距離センサアレイを設置し,PCと接続することのみを行い,膝には何も装着しない.
\section{利用イメージ}
\subsection{ワープロソフト利用時のマウスカーソル操作}
\subsecLabel{word}
ワープロソフトを利用する時,手はキーボード上にあることが多い.しかし,文字色や文字サイズの変更,図の挿入など,コマンド入力で行うことができない操作を行う時,マウスやタッチパッドに手を移動させ,操作する場面が多い.そこで,膝によってマウスカーソルを操作することで手の移動を削減することができる.
\img{htbp}{1.0}{usc1.pdf}{ワープロソフトの利用時に膝でマウスカーソルを操作するイメージ}{usc1}
\subsection{画面を見ながら作業するときの画面操作}
PCの画面を見ながら作業をする,例えば画面では回路図を写しながら,その回路図に従って電子回路を製作するときに,回路図をズームしたり別の回路図を見ようとすると,作業から手を離さなければいけない.そこで,膝を動かすことで手は作業に集中したまま画面の操作を行うことができる.\refImg{usc2}は
\img{htbp}{1.0}{usc2.pdf}{画面を見ながら作業をするイメージ}{usc2}
\section{本研究で行う操作}
本研究では膝の位置を認識することで,\refSubsec{word}で述べたようなマウスカーソル操作への応用を行う.これは,マウスカーソルのような繊細な操作を膝で行ったときの操作性や疲労感を明らかにし,膝による操作全体の改良を図るためである. 




\chapter{ユーザの膝の動作}\chaLabel{kneeMoving}
\section{概要}
マウスやタッチパッドの操作と異なり,膝は前方や後方に動かすことはできない.そのため,本研究で想定する膝の移動は,膝を傾けることによる左右方向と,かかとを浮かせたり床につけたりすることによる上下方向の移動である.\refImg{ov}は膝でマウスカーソルを操作するときのイメージである.ユーザは水色の矢印で示される方向に膝を動かし,システムは鉛直面における膝の2次元座標を計算する.

\img{htbp}{1}{sousa_overview.pdf}{膝によるマウスカーソル操作のイメージ}{ov}
\section{膝の移動方法}
\subsection{左右方向}
ユーザはマウスカーソルを左右に移動させたい時には,膝を左右に移動する.\refImg{sousalr1},\refImg{sousalr2}は足を正面から見たときの左右方向の移動のイメージである.\refImg{sousalr1}のように,このとき足先も同時に移動させ,足全体を移動させると,太ももに疲労が生じる.また,足に装置を取り付け,摩擦力を小さくするということも目的に反する.そこで\refImg{sousalr2}のように,左右方向に移動する時に限っては,足の位置をなるべく固定し,膝を左右に傾けることで移動することとした.
\img{H}{1.0}{lr1.pdf}{左右方向の操作イメージ1}{sousalr1}
\img{H}{1.0}{lr2.pdf}{左右方向の操作イメージ2}{sousalr2}
%\subsection{操作方法案1}
%マウスの操作では,カーソルを$x$軸方向の操作にはマウスを左右に動かす方法,$y$軸方向の操作にはマウスを\fixme{奥に押す},または手前に引くという動作を行う.これらを膝に置き換えた時,
\subsection{上下方向}
上方向に移動させたい時は,かかとを浮かせて膝を机に近づける.逆に下方向に移動させたい時は,足を手前に引き,その時に浮いたかかとを床に近づけることで,膝を机から遠ざける.\refImg{sousa1},\refImg{sousa3}は,足を横から見たときの上下方向の移動のイメージである.
\img{htbp}{1}{sousa1.pdf}{上下方向の操作イメージ1}{sousa1}
\img{htbp}{1}{sousa3.pdf}{上下方向の操作イメージ2}{sousa3}

\refImg{sousa1}に示されているような移動方法は,足が完全に地面についている時に,マウスカーソルは画面の一番下の位置になってしまう.そのため,ユーザが画面の真ん中付近にカーソルを移動する時にかかとを浮かせた状態を維持しなければならず,疲労が生じてしまう.そこで\refImg{sousa3}のように移動することで,足が完全に地面についている時にカーソルが真ん中にあるため,ユーザは比較的楽な姿勢で操作することができる.


%適用することができるが,膝を奥に動かしたり手前に引くということはできない.したがって,$y$軸方向の操作として,膝を持ち上げる動作を適用することとした.\refImg{sousa1}は$y$軸方向の操作方法として考案した膝の動かし方の1つである.画面の下の方へカーソルを動かす時には,かかとを下げ,カーソルが下限にくる時,かかとは完全に床につく状態になる.逆に画面の上の方へカーソルを動かす時には,かかとを浮かせる.


%しかし,この操作方法について意見を募ったところ,かかとを浮かせた状態で維持することが困難であり,著しく疲労を感じるという回答を受けた.特に,カーソルを画面の真ん中で維持することが困難であった.したがって,カーソルを画面の上下方向の真ん中に維持する時,膝の姿勢がユーザにとって特別な力を必要としない状態を取ることが望ましいことがわかった.
%\subsection{操作方法案2}
%操作方法案1の改善点を受け,操作方法案2を考えた.\refImg{sousa2}はそのイメージ図である.
%操作方法案2では足を手前に引き,かかとを浮かせた状態をカーソルが真ん中に来るものとする.カーソルを下に動かしたいときは,その位置からかかとを下げる.それに伴い,膝も下に下がる.カーソルを上に動かしたいときは操作方法案1と同様にかかとを更に浮かせる.
%操作方法案2についても意見を募った.しかし,カーソルを下方向に動かす動作は,足首が本来とは逆の方向に曲がるために,正しい操作が可能かはその人の足首の柔らかさに依存した.したがって,\fixme{足首に負担がかからない動作方法}について,操作方法案を再考することとした.
%\subsection{操作方法案3}
%操作方法案2の改善点を受け,操作方法案3を考えた.\refImg{sousa3}はそのイメージ図である.
%操作方法案3では,かかと含め,足が完全に床についている状態をカーソルが画面の真ん中に来るものとする.カーソルを下に動かしたいときは,足を十分引き膝の位置を下げる.カーソルを上に動かしたいときは操作方法案1と同様にかかとを浮かせる.
%操作方法案3について意見を募ったところ,操作方法案1で問題になった疲労感,操作方法案2で問題となった足首に対する依存が解消されたとの回答を受けた.したがって,この操作方法でカーソル操作を行うものとする.
\chapter{膝位置認識とカーソル座標計算を行う\\プロトタイプの実装}\chaLabel{implementation}
\section{概要}
本章では,膝の位置を認識し,マウスカーソルに適用するプロトタイプの実装について述べる.プロトタイプは,三角法を用いた光学式距離センサ10個を一列に並べたセンサアレイ,マイコン,パーソナルコンピュータからなるハードウェアと,センサから距離データを取得し,フィルタ処理,膝の位置の計算を行うソフトウェアからなる.\refImg{proto_outline}に,プロトタイプの概要を示す.
\img{htbp}{1.0}{proto_outline.pdf}{プロトタイプの概念図}{proto_outline}

プロトタイプによる膝の位置認識と,カーソル座標への変換の流れは以下のようになっている.
\begin{enumerate}
	\item ユーザは机の裏に距離センサアレイを設置し,パーソナルコンピュータと接続する.そして,膝を上下左右に動かす.
	\item 距離センサアレイが,各センサと膝との距離を計測する.
	\item 全ての距離センサの計測値は膝位置計算ソフトウェアに送信される.
	\item ソフトウェアは,膝の位置を計算し、カーソル座標を計算する.\\ソフトウェアの詳細な流れを以下に示す.
	\begin{enumerate}
		\item 計測値を指数移動平均フィルタにかける.
		\item 机に垂直な面において膝がどの位置にあるかを計算する.
		\item キャリブレーションにより,ユーザの膝が移動できる範囲を記録する.
		\item キャリブレーション時の値とマウスカーソルを描画するディスプレイの解像度からカーソル座標を計算する.
	\end{enumerate}
\end{enumerate}
この時ユーザは机の裏に距離センサアレイを設置し,パーソナルコンピュータとの接続のみを行い,膝には何も装着しない.

\section{ハードウェア}\secLabel{proto1}
距離センサはSHARP GP2Y0E03\footnote{http://www.sharp.co.jp/products/device/lineup/selection/opto/haca/diagram2.html}を使用した.この距離センサは三角測量の原理を用い,対象までの距離を計測する.個々のセンサは,スレーブアドレスが初期値(0x40)で統一されているために,アプリケーションノート\footnote{http://www.sharp.co.jp/products/device/doc/opto/gp2y0e02\_03\_appl\_j.pdf}に記載されているe-fuseプログラミングの手順で,スレーブアドレスの変更を行なった.
\SysName では,長さ約30\si{cm}のプラスティック製の定規に,両面テープでセンサ本体を30\si{mm}間隔で定規に固定し,配線類はセロハンテープで固定した.
%30\si{mm}という間隔は,センサアレイが膝が左右に動く範囲を全てカバーできる範囲として設定した.
\refImg{im1}は実際に製作したプロトタイプの一部である.
距離センサは横向きにして1列に並べた.これは,GP2Y0E03のアプリケーションノート\footnotemark[2]には,センサを物体の移動方向が\refImg{im2}の黒い矢印の方向ではなく,赤い矢印の方向になるように設置する方が,誤差が少ないためである.
\par
個々の距離センサはArduino MEGA 2560(Arduino)ユニバーサル基板を介して接続される.\iic 通信を用いて,Arduinoが距離センサの値を読み取る.ArduinoはパーソナルコンピュータとUSBで接続され,シリアル通信を用いて,10個の距離データをひとまとめにしたものを1フレームとして,Arduinoからパーソナルコンピュータへ送信する.
\img{htbp}{0.95}{proto.pdf}{製作したプロトタイプの一部}{im1}
\img{htbp}{0.5}{proto2-1}{移動物体に対する距離センサの設置方向}{im2}
\section{膝の位置の計算} 
膝の位置の計算には,
%スマートウォッチに搭載した距離センサから指の位置をトラッキングした
Xiaoら\cite{Xiao:2018:LOP:3173574.3173669}の方法を参考にした.ソフトウェアでは,以下の4点を行う.
\begin{itemize}
	\item シリアル通信で受信した距離データを指数移動平均フィルタにかける.
	\item 膝の位置を計算する.
	\item キャリブレーションを行う.
	\item 膝の位置をマウスカーソルの座標に変換する.
\end{itemize}
プログラム言語はPythonを用いた.シリアル通信のためのライブラリとしてPySerial,ポインタを描画するGUIのためのライブラリとしてPyQtを用いた.
\subsection{指数移動平均フィルタ}
指数移動平均フィルタは,時間の経過とともに重みを指数関数的に減少させるフィルタである.最新のデータを重視する一方で,古いデータについても考慮に入れるという特徴を持つ.重みの減少度合いは平滑化係数$\alpha$を用いて表される.本プロトタイプでは,調整の結果$\alpha = 0.1, 0.65$の2つを使用している.時間$t$の距離データ$s^t$に対しフィルタリング後の値$D^t$を以下のように計算し,フィルタを実装した.
\begin{eqnarray}
	D^t = \alpha (s^t - D^{t-1}) + D^{t-1}
\end{eqnarray}
\subsection{膝位置認識}
\subsecLabel{acq}
時間$t$における膝の位置$(K^t_x, K^t_y)$を次のように計算する.
\begin{enumerate}
	\item 各距離センサの値を$\alpha=0.1$の指数移動平均フィルタにかける.これを$D^t_i$と表す.ただし,$i$は距離センサの番号を表し,\refImg{im1}の一番左の距離センサから順に$i=0,1,..9$と番号を振ることとする.
	\item $K^t_y$を$D^t_i $の最小値とする.
		\begin{eqnarray}
		 	K^t_y = \min_i(D^t_i)
		 	\label{formula:f1}
		\end{eqnarray}
	\item $i$番目の距離センサについて,重み$w_i$を式\ref{formula:f2}のように計算する.ここで,$d$は重み調整の定数である.\SysName では調整の結果$d=2$としている.
		\begin{eqnarray}
			w_i = \cfrac{1}{D^t_i - K^t_y + d}
		\label{formula:f2}
	\end{eqnarray}
	\item $w_i$から,$K^t_x$を計算する.
		\begin{eqnarray}
		 	 K^t_x=\cfrac{\sum_i iw_i}{\sum_i w_i}
		 	\label{formula:f3}
		\end{eqnarray} 
	\item 得られた$(K^t_x, K^t_y)$を$\alpha=0.65$の指数移動平均フィルタにかける.
\end{enumerate}
\section{膝の位置からカーソル座標への変換}
\subsection{キャリブレーション}
\subsecLabel{cal}
\refSubsec{acq}で行なった,時間$t$の膝の位置をマウスカーソルの座標に変換するために,キャリブレーションとしてユーザの膝の位置を記録する.キャリブレーションは,ユーザが膝を動かす上下左右の限界点と,マウスカーソルがディスプレイの中心にある時の膝の位置の5点で行う.
%膝の位置の計算には指数移動平均フィルタを使っているため,最初の数フレームは実際とは異なる値が出力されてしまう.そのため,
得られた上下左右および真ん中の点を$(C_{upper}, C_{lower}, C_{left}, C_{right}, (C_{center_x}, C_{center_y}))$と表す.
\subsection{カーソル座標への適用}
\refSubsec{acq}で計算された時間$t$における膝の位置$(K^t_x, K^t_y)$を,\refSubsec{cal}で得たキャリブレーションをもとに,パーソナルコンピュータのディスプレイ上のカーソル座標$(P^t_x, P^t_y)$へと次のように変換する.なお,ここでは解像度が$(W_x, W_y)$のディスプレイを想定している.
\begin{eqnarray}
	P^t_x = 
	\begin{cases}
		\cfrac{(K^t_x - C_{left}) \left( \cfrac{W_x}{2} \right)}{C_{center_x} - C_{left}} & (K^t_x < C_{center_x})\\
		\cfrac{(K^t_x - C_{center_x}) \left( \cfrac{W_x}{2} \right)}{C_{right} - C_{center_x}} & (C_{center_x} \leq K^t_x)
	\end{cases}	 
\end{eqnarray}
\begin{eqnarray}
	P^t_y = 
	\begin{cases}
		\cfrac{(K^t_y - C_{upper}) \left( \cfrac{W_x}{2} \right)}{C_{center_y} - C_{upper}} & (K^t_y < C_{center_y}) \\
		\cfrac{(K^t_x - C_{center_y}) \left( \cfrac{W_x}{2} \right)}{C_{lower} - C_{center_y}} & (C_{center_y} \leq K^t_y)
	\end{cases}
\end{eqnarray}


%\subsection{プロトタイプ1の性能評価}\subsecLabel{proto1_problem}
%実装を行なったプロトタイプを机の裏に設置し,自由なポインタの操作ができるか試みた.キャリブレーション次第では膝を静止させた時にポインタも静止するが,多くの場合は膝を静止させているにも関わらずポインタは左右に振れるなどした.センサからの値を観察したところ,\fixme{膝がかかっていない部分の値}が激しく上下していることがわかった.原因として,以下のようなものが挙げられた.

%\section{プロトタイプ2}
%\subsection{改良点}
%\refSubsec{proto1_problem}であげた問題を解決するために,プロトタイプに改良を行った.
%\begin{itemize}
	%\item{改良1: } 問題点1.について,すべてのセンサを90度左回転させて設置した.また同時に問題点4.について,
	%\item{改良2: } 問題点2.について,実装を\fixme{ブレッドボードでの接続からユニバーサル基板を用いて配線を固定すること}とした.
	%\item{改良3: } 問題点3.について,センサの直下には床に白紙を貼り付けることで解決を図った.
%\end{itemize}

%\subsection{改良後の評価}
%改良1を加えたプロトタイプ2を同様に机の裏に設置し,自由なポインタの操作ができるか実験した.プロトタイプ1ではポインタの左右の振れ方はとても大きく,キャリブレーション次第では操作もままならないほどであったが,改良1により軽減された.問題点1.の指摘は正しく,改良後のセンサの配置方向が正しいと考えられる.しかし,依然としてポインタが左右に振れた.
%\par
%改良2を加えて同様の実験を行なったところ,ポインタの左右の振れの改善はあまり見られなかった.これは,\fixme{センサと膝は直接触れることはないため,設置した状態から動くことはないからであると考える.}しかし,ユニバーサル基盤上に実装したことで,接続部分がブレッドボードよりも薄くなり机裏への設置が容易になった.加えてプロトタイプ1では作業中に簡単に配線が抜けてしまうことがあったが,配線が固定され抜けることがなくなった.したがって,ブレッドボードの実装に戻すことはしなかった.
%\par
%改良3を加えて,まずセンサの値を観察したところ,相変わらずノイズは観察されるが,指数平均平滑フィルタを通した後の値は改良3を加える前よりも安定した.床面を白くすることで赤外線の反射が多くなり,正しく測定できているからであると考える.


\chapter{膝によるマウスカーソル操作の性能評価} \chaLabel{exp}
本章では,製作したプロトタイプを用いて膝を用いたマウスカーソル操作の特徴を実験を通して調査する.
\section{目的}
本実験では,膝によるマウスカーソル操作をフィッツの法則\cite{fitts}に当てはめて,その性能を明らかにする.またアンケートから,ユーザの操作性,疲労感を明らかにする.
\section{評価方法}
実験の評価は,フィッツの法則を用いて行う.フィッツの法則は,式\ref{formula:fitts}によって表される.
\begin{eqnarray}
	MT = a + b\log_2{(D/W + 1)}
	\label{formula:fitts}
\end{eqnarray}
式\ref{formula:fitts}に用いられている各係数は以下の通りである.
\begin{itemize}
	\item {$MT$(Moving Time): }ターゲットを選択するまでにかかる時間
	\item {$a,b$: }ユーザと装置に依存する定数
	\item {$D$: }ポインタがある場所からポインティングするターゲットまでの距離(ターゲット間距離)%ここではターゲットが配置されている円の直径に近似する
	\item {$W$: }選択するべきターゲットの幅
	\item { $\log_2{(D/W + 1)}$ [bit]: } 課題の困難度を表す数値\ (Index of Difficulty(ID)と呼ばれる.)
\end{itemize}
IDが高くなればなるほど,ポインティングが難しくなり,MTも大きくなる.
性能の評価にはIDから課題を達成するのに要した時間を割った値(Throuput, TP)が用いられる
%(式\ref{formula:tp})
.
\begin{comment}
	\begin{eqnarray}
	TP = \cfrac{ID}{MT}
	\label{formula:tp}
\end{eqnarray}
\end{comment}

\section{実験手順}

実験には,ISO9241-411\cite{9241411}に記載されている,マルチディレクショナルポインティングタスクに基づいて製作したプログラムを使用した.\refImg{mdpt}は実験に使用したプログラムの動作イメージである.

参加者は円周上に配置された13個のターゲットを,0から13の順に選択する.選択するべきターゲットは水色で示され,それ以外のターゲットは背景と同じ色で表される.ターゲットを1回選択することを1試行と数え,はじめの0番のターゲットの選択を除いた13試行を1タスクと数える.
膝を動かしてカーソルを移動させ,ターゲットとポインタが重なった時に選択操作を行う.本プログラムでは,選択操作は足ではなくキーボード上のEnterキーで行う.
\img{htbp}{0.7}{mdpt.png}{実験に使用したプログラム}{mdpt}

実験条件として,ターゲット幅($W$)とターゲット間距離($D$)を次のように変化させた.

\begin{itemize}
	\item $D$: 2.0,5.0,8.0 (インチ)
	\item $W$: 0.5,1.0,1.5 (インチ)
\end{itemize}
これにより得られる,以下の9つのIDの条件を1タスクずつ行う.これを1セッションと数える.
\begin{itemize}
	\item $ID = $: \{1.22, 1.59, 2.12, 2.32, 2.59, 2.66, 3.17,  3.46, 4.09\}
\end{itemize}
実験は両膝について3セッションずつ行うものとし,片膝について行うことを1ピリオドと数える.したがって,参加者1人につき,2ピリオド(左右の膝)$\times$13(ターゲット数)$\times$9(条件)$\times$3(セッション)=702試行を行う.
1セッション終了ごとに3分間,1ピリオド終了後に10分間の休憩時間を設けた.また,2ピリオド終了後にアンケートを行った.
%アンケートは操作の使いやすさ,快適さ,スムーズさ,肉体的難しさ,精神的難しさと腹部,太もも,ふくらはぎ,足の疲労感を5点リッカート尺度で評価した.
本実験には3名が参加した.全員が男性であり,年齢はそれぞれP1:22歳,P2:24歳,P3:22歳である.P1,P3は左膝・右膝,P2は右膝・左膝の順でそれぞれ実験を行なった.
参加者はセッションの開始前にプロトタイプを設置した机の前に座り,椅子の高さ,ディスプレイからの距離を参加者の好みに合わせて調節した(机の高さは固定である).椅子の位置は,プロトタイプが設置されている机の位置を表す黒色テープを基準に調整した.その後,膝を動かすことができる範囲を決定するためのキャリブレーションを行う.実験全体では6回キャリブレーションを行う.ピリオドの最初のセッションでは,キャリブレーション後に練習時間を5分設けた.
\section{実験機器}
実験には,プロトタイプ,ディスプレイ(解像度:1920*1080ピクセル,21.5インチ,DELL社製ST2220Lb),ディスプレイを設置する台座,キーボード(Apple社製,Magic Keyboard),
%マウス(Apple社製,Magic Mouse)
を使用した.
%マウスは,キャリブレーションを行うときに実験者のみが使用し,実験時には使用しない.
実験条件の$D$と$W$の数値は,実験プログラムの中で自動的にディスプレイのピクセル密度(102.42[ppi])を元にインチからピクセルに変換している.実験プログラムはディスプレイに映し出され、参加者はディスプレイを見てキーボードとプロトタイプを操作する.\refImg{jikken_image}は,実際の実験の様子である.
\img{htbp}{1.0}{jikken_image.pdf}{実験の様子}{jikken_image}

\section{収集データ}
解析のために収集したデータは次のとおりである.
\begin{itemize}
	\item 試行ごとのターゲットの選択に要した時間
	\item その試行でターゲット選択が正しくできたかを表すフラグ
\end{itemize}


\section{実験結果}
\refImg{exp_left}は左膝の,\refImg{exp_right}は右膝の実験結果を表す.横軸は式\ref{formula:fitts}におけるID,縦軸は選択時間であり,グラフには各参加者の選択時間と,選択時間を元に線形回帰で求めた直線が描かれている.
\refImg{exp_error}は両膝のエラー率を表す.横軸は参加者であり,縦軸はエラー率が百分率で表される.グラフにはセッションごとにP1,P2,P3のエラー率が棒グラフで表され,セッションごとの参加者間の平均エラー率が折れ線グラフで表されている.全参加者のエラー率の平均は,左膝で1.14\%,右膝で1.71\%であった.\refImg{exp_tp}は参加者ごとに左膝,右膝のスループットを計算した結果である.横軸は参加者であり,縦軸はスループットである.グラフにはセッションごとにP1,P2,P3のスループットが棒グラフで表され,セッションごとの参加者間の平均スループットが折れ線グラフで表されている.全体のスループットの平均は,左膝で$1.497$[bit/s],右膝で$1.540$[bit/s]であった.t検定を行なったところ,左右の膝のスループットに有意な差はなかった($t=-0.151, df=4$).
表\ref{tab:anche}にアンケートの結果を示す.操作の使いやすさ,精神的な難しさ,腹部の疲労感,ふくらはぎの疲労感で比較的良好な結果を得た.
%今回の実験では,全参加者のエラー率の平均が,左膝で1.14\%,右膝で1.71\%であり,これはVellosoら\cite{velloso:hal-01599657}やHorodniczyら\cite{Horodniczy:2017:FHE:3025453.3025625}が行なった実験よりも低い値を得た.
\img{htbp}{0.9}{left.pdf}{左膝の選択時間とそのモデル}{exp_left}
\img{htbp}{0.9}{right.pdf}{右膝の選択時間とそのモデル}{exp_right}
\img{htbp}{0.9}{error.pdf}{両膝のエラー率}{exp_error}
\img{htbp}{0.9}{tp.pdf}{両膝のスループット}{exp_tp}
\begin{table}[htbp]
	\begin{center}
		\caption{アンケートの結果}
		\begin{tabular}{|c|c|c|c|c|}
		\hline 
	   		& P1 & P2 & P3 & 平均   \\ \hline
			\begin{tabular}{c}操作の使いやすさ\\ (1: 使いにくい - 5:使いやすい)\end{tabular}  & 4  & 4  & 3  & 3.67 \\ \hline
			\begin{tabular}{c}操作の快適さ\\ (1: 快適でない- 5:快適である)\end{tabular}  & 4  & 3  & 3  & 3.33 \\ \hline
			\begin{tabular}{c}操作のスムーズさ\\ (1: スムーズでない - 5:スムーズである)\end{tabular}  & 4  & 3  & 3  & 3.33 \\ \hline
			\begin{tabular}{c}肉体的な難しさ\\ (1: 簡単である - 5:難しい)\end{tabular}  & 3  & 1  & 4  & 2.67 \\ \hline
			\begin{tabular}{c}精神的な難しさ\\ (1: 簡単である - 5:難しい)\end{tabular}  & 2  & 1  & 2  & 1.67 \\ \hline
			\begin{tabular}{c}腹部の疲労感\\(1: 疲れていない - 5:疲れている)\end{tabular}& 1  & 1  & 1  & 1.00 \\ \hline
			\begin{tabular}{c}太ももの疲労感\\(1: 疲れていない - 5:疲れている)\end{tabular}& 4  & 1  & 3  & 2.67 \\ \hline
			\begin{tabular}{c}ふくらはぎの疲労感\\(1: 疲れていない - 5:疲れている)\end{tabular}& 3  & 1  & 2  & 2.00 \\ \hline
			\begin{tabular}{c}足の疲労感\\(1: 疲れていない - 5:疲れている)\end{tabular}& 4 & 1  & 3  & 2.67 \\ \hline
		
		\end{tabular}
		\label{tab:anche}
	\end{center}
\end{table}

\section{考察}
\subsection{エラー率とキャリブレーション}
\refCha{exp}で行なった実験の結果,膝による操作での選択ミス率は左膝で1.14\%,右膝で1.71\%であり,Horodniczyら\cite{Horodniczy:2017:FHE:3025453.3025625}やVellosoら\cite{velloso:hal-01599657}の手法の結果よりも低い値であった.原因として以下の3点が考えられる.

1点目はキャリブレーションを1セッションごと,計6回と多く行なっていたことである.このため,キャリブレーションの回数を減らすことでエラー率やスループットにどのような影響を与えるかを調査する必要がある.
2点目は膝はカーソル操作のみを行い,選択操作はキーボードのEnterキーで行なっていたことである.したがって,マウスの機能をすべて膝と足のみを使って実現した場合,エラー率が変化する可能性がある.
3点目はポインティングタスクの難易度である.今回の実験では,IDの最大は4.09[bit]であったが,最大5.67[bit]で行う実験も存在した\cite{Horodniczy:2017:FHE:3025453.3025625, velloso:hal-01599657}.したがって,よりIDの高い条件での実験を行う必要がある.

\subsection{操作性と疲労感の改善}

\refCha{kneeMoving}にて,疲労感を考慮して膝の移動方法を設計した.しかし太ももと足の疲労は比較的疲労感が高い.今後膝の移動方法をさらに見直し,疲労感を改善する工夫が必要である.さらに,操作の快適さ,スムーズさにも改善の余地があるとわかる.ハードウェア設計の改善を行うことで,操作性についても改善を図る.





\chapter{今後の展望}\chaLabel{discussion}
本章では,プロトタイプの既知の問題点に対する今後の展望について述べる.

今回のプロトタイプの設計は,直定規に距離センサを貼り付けたため,直線的な形状となった.しかし,膝を左右に傾ける時,膝の運動は直線運動ではなく円弧に近い運動をする.そのため膝がハードウェアの両端付近にある場合と中心付近にある場合とでは,ユーザが水平にカーソルを移動するとき,上下にカーソルがずれてしまう問題が発生する.加えて,ユーザの足の長さによって描く円弧の半径は異なると考える.このことから,膝の運動のユーザ間での違いについてより詳細な分析を行い,その結果からハードウェアの形状を再設計する必要がある.

%ハードウェアは直線形状ではなく円弧に近い形にする必要があり,さらにその形状も,ユーザによって簡単に変えることができるような設計が求められる.そのために,膝の運動のユーザ間での違いについてより詳細な分析も必要となる.

また,今回のプロトタイプでは距離センサを10個をそれぞれ30\si{mm}の間隔で配置した.この設計について,距離センサの数や配置の間隔を増減させた時にも同様なマルチディレクショナルポインティングタスクの実験を行うことで,最適な設計を導く.
\chapter{結論}\chaLabel{conclusion}
本研究では特別な装置を足に装着することなく,かつ簡単な設置方法で足によるコンピュータの操作を可能にすることを目的とし,アプローチとして机の裏に設置した距離センサアレイとArduinoからなるハードウェアと,膝の位置を認識するソフトウェアからなるプロトタイプを製作し,膝の位置を認識するためのプログラムを実装した.また,膝の位置をコンピュータ上のマウスカーソルの位置に反映させ,膝を用いたマウスカーソル操作に適用した.特徴や性能を調査するために,ISO9241-411に準拠したマルチディレクショナルポインティングタスクを用いた実験を行い,フィッツの法則モデルを示した.今後は膝の運動の調査とそれに合わせたハードウェアの設計,条件やキャリブレーションの回数を変化させた実験などを行い,さらなる操作性の向上を目指す.
\chapter*{謝辞}
\addcontentsline{toc}{chapter}{\numberline{}謝辞}
本論文の執筆に際し,指導教員である高橋伸准教授と志築文太郎准教授には多大なご助力を賜り,深く感謝を申し上げます.特に高橋伸准教授には研究の方針や内容について多くの指針やアドバイスを頂きました.心より御礼申し上げます.また,インタラクティブプログラミング研究室の皆様には,研究や研究室生活において様々なご助言をいただきました.特にUBIQUITOUSチームの皆様にはチームゼミをはじめ,設計や論文執筆時の添削など多くの点でご支援をいただき,深く感謝を申し上げます.最後に,研究室生活を支えてくださった家族,友人,研究においてお世話になった方々に感謝を申し上げます.




%図表には番号と説明(caption)を付け,文章中で参照する.表
%\ref{table:fundamental_data_type}と図\ref{figure:sample}はそれぞれ表と図
%の例である.表の説明は上に,図の説明は下に書くことが多い.図の挿入に用い
%るパッケージについては使用環境に合わせて自由に選択してほしい.

%\begin{table}[hbt]
%\caption{表の例}
%\label{table:fundamental_data_type}
%\begin{center}
%\begin{tabular}{| c | r | r | r | r |}
%\hline
%年 度 & 1年次 & 2年次 & 3年次 & 4年次 \\
%\hline
%1995 & 85 & 92 & 86 & 88 \\
%1996 & 83 & 89 & 90 & 102 \\
%1997 & 88 & 87 & 91 & 112 \\
%1998 & 144 & 93 & 90 & 115 \\
%\hline
%\end{tabular}
%\end{center}
%\end{table}
%\medskip

%\begin{figure}[htbp]
%\begin{center}
%\includegraphics[width=3cm]{sample.eps}
%\psfig{file=sample.eps,scale=0.6}
%\epsfile{file=sample.eps,scale=0.6}
%\end{center}
%\caption{図の例}
%\label{figure:sample}
%\end{figure}





\newpage

\addcontentsline{toc}{chapter}{\numberline{}参考文献}
\renewcommand{\bibname}{参考文献}




%% 参考文献に jbibtex を使う場合
\bibliographystyle{junsrt}
\bibliography{ref}
%% [compile] jbibtex sample; platex sample; platex sample;

%% 参考文献を直接ファイルに含めて書く場合
%\begin{thebibliography}{1}
%\bibitem{RakRak}
%野寺隆志.
%\newblock 楽々 \LaTeX.
%\newblock 共立出版, 1990.

%\bibitem{JiyuuJizai}
%磯崎秀樹.
%\newblock \LaTeX 自由自在.
%\newblock サイエンス社, July 1992.

%\bibitem{bryant-ieeetc86}
%Randal~E. Bryant.
%\newblock Graph-based algorithms for {B}oolean function manipulation.
%\newblock {\em IEEE Transactions on Computers}, Vol. C-35, No.~8, pp. 677--691,
%  August 1986.
%\end{thebibliography}
\appendix{}
\chapter{実験後アンケート用紙}
実験後のアンケートに用いた用紙を以下に示す.
\begin{figure}[htbp]
 \begin{center}
  \fbox{\includegraphics[width=0.9\hsize, angle=0]{./img/anquette.pdf}}
 \end{center}
 \caption{実験後アンケート用紙}
 \label{fig:one}
\end{figure}

\end{document}
