\chapter{序論}

\section{背景}
パーソナルコンピュータ、スマートフォン、タブレット端末に代表される情報端末が普及し、それらの操作方法は多岐にわたる。一般的に、パーソナルコンピュータはマウスやタッチパッドなどを用い、スマートフォン、タブレット端末においてはタッチパネルを搭載しており、直接触れることで、コンテンツを選択することができる。しかし、これらの操作方法は手を用いることを前提としており、特にパーソナルコンピュータの操作においてはキーボードも同時に用いるため、マウスと同時に操作することはできない。こうした問題を解決するために、手を用いない操作手法が多数提案されている。\\
\\
\TODO{視線や音声など、具体的に挙げる}\\
\\
本研究では、通常パーソナルコンピュータの使用には関わらない、足を用いた操作手法に着目する。具体的には、足をマウス操作に用いることで、上記の問題の解決をはかる。
\\
\\
\TODO{従来研究では足に何かを装着しなければならない点を挙げる}




\section{目的・アプローチ}


\section{貢献}
\section{本論文の構成}
