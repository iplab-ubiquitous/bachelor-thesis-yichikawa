\chapter{序論}\chaLabel{background}

\section{背景}
自動車のアクセルペダルやブレーキペダル,ピアノやオルガンのペダルに代表されるように,我々は日常的に足による操作を行なっている.しかし,パーソナルコンピュータやスマートフォンを操作する際,我々は手を中心に操作を行い,足に操作が割り当てられることはない.足による操作を用いたコンピュータ向けインタフェースの研究は,1960年代から存在している\cite{1698228}が,現在は手による操作が中心である.しかし,Multitoe\cite{Augsten:2010:MHI:1866029.1866064}のようなタッチ認識可能な床面とのインタラクションや,手がふさがった状態におけるモバイル機器の操作\cite{Fan:2017:ESF:3123021.3123043, okumura_2011}など,足によるインタラクションの研究は関心が高まっている.\par
机上でコンピュータを用いる作業では両手をキーボードの操作に充てることが多い.足によってマウスカーソルの操作を行うことで,両手はキーボード操作を行ったまま,マウスカーソルの制御が可能になる.そのため,足を使ってマウスカーソルの操作を行う研究が盛んである.
%
そのアプローチには,装置を足で動かす方法\cite{Pearson:1986:MMD:22627.22392, Pearson:1988:EET:49108.1046356}や,足の位置によって摩擦力を変えることができる機構を取り付けた靴\cite{Horodniczy:2017:FHE:3025453.3025625}を用いた手法などがあるが,これらは体の一部に装置を取り付けるあるいは大型な装置を用いるものであるため,ユーザの衣服などに制限が生じる,装置を持ち運びができないなどの制約が存在する.
%このような物理的制約を不要にする研究は少ない\cite{velloso:hal-01599657}.



\section{目的・アプローチ}
本研究の目的は,特別な装置を足に装着することなく,かつ小型な装置で,足を用いたコンピュータの操作を可能にし、前節で述べた問題を解決することである.そのアプローチとして,机下に取り付けた装置から膝の動作を読み取る.膝を使う理由は2点ある.まず,膝は足先と比べてより自由な操作が行える可能性がある.さらに,先行研究には膝をコンピュータへの入力として用いる例は少ない\cite{1698228}ため,膝と足による操作の組み合わせによりさらなるインタラクションの拡張が可能である.

本研究では膝の位置を認識し,マウスカーソルの操作に適用する.膝の位置の認識には市販の距離センサを10個並べた距離センサアレイと,マイコン,膝の位置を計算するソフトウェアからなるプロトタイプを用いる.ユーザは膝を上下左右に動かすことで,マウスカーソルの操作を行うことができる.
%膝の動作を読み取るシステムの開発を行い,膝を使ってポインタ操作を行う手法を提案する.ユーザは机の下で膝を動かすことで,PC上のポインタを操作することができる.システムのハードウェアには,距離センサを用いた.
%距離センサを用いたトラッキング技術には,例えばAIRBAR\cite{AIRBAR}やLumiwatch\cite{Xiao:2018:LOP:3173574.3173669}が挙げられるが,膝に適用した例はない.また,
%カメラを用いないことで,設置や構築が容易であるという利点があるために,カフェのテーブルや,ホテルのデスクのように様々な場所で設置,利用可能である.\\

\section{貢献}
本研究の貢献を以下に述べる.
\begin{itemize}
	\item ユーザの疲労感を考慮した膝の移動方法を提案した.
	\item 距離センサ10個を用いたプロトタイプを製作し,膝の位置を認識し,マウスカーソル操作へ応用した.
	\item プロトタイプを用いてフィッツの法則に基づく実験を行い,その結果から今後の設計に関する課題を明らかにした.
\end{itemize}
\section{本論文の構成}
\refCha{background}では,本研究の背景,目的とそれに対するアプローチについて述べた.\refCha{relatedwork}では,本研究に関連する研究について述べる.
%\refCha{system}では,膝位置認識システムの概要と
\refCha{kneeMoving}では,本研究で提案する膝の認識におけるユーザの膝の動かし方と、膝位置認識システムの利用イメージついて述べる.\refCha{implementation}では,膝の位置を認識し,マウスカーソルの座標に反映するためのプロトタイプの実装について述べる.\refCha{exp}では,プロトタイプを用いて行なった,フィッツの法則に基づいた実験とその結果について述べる.\refCha{discussion}では,プロトタイプについて現在明らかになっている問題点とそれに対する改善の方針を述べる.\refCha{conclusion}では,本研究の結論を述べる.

\begin{comment}
	\fixme{
	\begin{itemize}
		\item 本研究では膝によるマウスカーソル操作を調査する?
		\item 足を使ってみたい
		\item 足先の研究しかない,膝と組み合わせることで様々なインタラクションが可能になる
		\item 膝使ったものは少なく問題がある
		\item 膝を使ったことの理由→足の既存手法と組み合わせることができる,膝の可動域が広い(が先行研究が少ない
		\item いずれにせよ,関連研究が終わるまでに「膝によるマウスカーソル操作」という話に落とし込む
	\end{itemize}
}
\end{comment}
