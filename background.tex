\chapter{序論}\chaLabel{background}

\section{背景}
自動車のアクセルやブレーキペダル,ピアノやオルガンのペダルに代表されるように,我々は日常的に足による操作を行なっている.しかし,パーソナルコンピュータをデスクトップ上で作業をする際,我々は手を中心に操作を行い,足に操作が割り当てられることはない.足による操作を用いたコンピュータ向けインタフェースの研究は,1960年代から存在している\cite{1698228}が,現在は手による操作が中心である.しかし,Multitoe\cite{Augsten:2010:MHI:1866029.1866064}のようなタッチ認識を可能にした床面とのインタラクションや,手がふさがった状態におけるモバイル機器の操作\cite{Fan:2017:ESF:3123021.3123043, okumura_2011}など,新たに足によるインタラクションの研究は関心が高まっている.\par
その中でも,デスクトップでの特にコンピュータを使った作業中は両手をキーボードの操作に充てることが多い.そのため,足によるマウスのようなカーソルやポインタの操作を行う研究が盛んである.
%足によってマウスのような操作を行うことで,両手は他の操作を行ったまま,マウスポインタの制御が可能になる.
そのアプローチは,装置を足で動かす方法\cite{Pearson:1986:MMD:22627.22392, Pearson:1988:EET:49108.1046356},可変摩擦機構を取り付けた靴\cite{Horodniczy:2017:FHE:3025453.3025625}があるが,これらは体の一部に装置を取り付けるあるいは大型な装置を用いるものであるため,衣服などに制限が生じたり持ち運びができないなどの制約が加わってしまう.
%このような物理的制約を不要にする研究は少ない\cite{velloso:hal-01599657}.



\section{目的・アプローチ}
前節で述べた問題を解決するために,本研究の目的は,特別な装置を足に装着することなく,かつ簡単に,足を用いたコンピュータの操作を可能にすることである.そのアプローチとして,机下に取り付けた装置から膝の動作を読み取る.膝は足と比べて動かす時に動作が大きくなりにくく,足元よりも動かせる範囲が大きいと考える.また,先行研究では足先を用いる操作が多いのに対し,膝を使うものは少ない\cite{1698228}ため,膝と足による操作の組み合わせによりさらなるインタラクションの拡張が可能である.

本研究では膝の位置を認識し,マウスカーソルの操作に適用する.膝の位置の認識には市販の距離センサを用い,これを10個並べた距離センサアレイと,Arduino,ソフトウェアからなるプロトタイプを製作する.ユーザは距離センサアレイを机の裏に設置し,机の下で膝を上下左右に動かすことでマウスカーソルの操作を行うことができる.
%膝の動作を読み取るシステムの開発を行い,膝を使ってポインタ操作を行う手法を提案する.ユーザは机の下で膝を動かすことで,パーソナルコンピュータ上のポインタを操作することができる.システムのハードウェアには,距離センサを用いた.
%距離センサを用いたトラッキング技術には,例えばAIRBAR\cite{AIRBAR}やLumiwatch\cite{Xiao:2018:LOP:3173574.3173669}が挙げられるが,膝に適用した例はない.また,
%カメラを用いないことで,設置や構築が容易であるという利点があるために,カフェのテーブルや,ホテルのデスクのように様々な場所で設置,利用可能である.\\

\section{貢献}
本研究の貢献を以下に述べる.
\begin{itemize}
	\item 特別な装置を装着することがない膝の位置認識の提案と,膝によるマウスポインタ操作への応用を示した.
	\item 距離センサ10個を用いたプロトタイプを製作し,膝の位置を認識し,マウスカーソル操作へ応用した.
	\item プロトタイプを用いて,フィッツの法則に基づく実験を行なった.
\end{itemize}
\section{本論文の構成}
\refCha{background}では,本研究の背景,目的とそれに対するアプローチについて述べた.\refCha{relatedwork}では,本研究に関連する研究について述べる.\refCha{system}では,本研究で提案する膝の認識におけるユーザの膝の動かし方について述べる.\refCha{implementation}では,膝の位置を認識し,マウスカーソルの座標に反映するためのプロトタイプの実装について述べる.\refCha{exp}では,プロトタイプを用いて行なった,フィッツの法則に基づいた実験とその結果について述べる.\refCha{discussion}では,プロトタイプについて現在わかっている問題点とそれに対する改善の方針を述べる.\refCha{conclusion}では,本研究の結論を述べる.

\begin{comment}
	\fixme{
	\begin{itemize}
		\item 本研究では膝によるマウスカーソル操作を調査する?
		\item 足を使ってみたい
		\item 足先の研究しかない,膝と組み合わせることで様々なインタラクションが可能になる
		\item 膝使ったものは少なく問題がある
		\item 膝を使ったことの理由→足の既存手法と組み合わせることができる,膝の可動域が広い(が先行研究が少ない
		\item いずれにせよ,関連研究が終わるまでに「膝によるマウスカーソル操作」という話に落とし込む
	\end{itemize}
}
\end{comment}
