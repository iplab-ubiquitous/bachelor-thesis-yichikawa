\chapter{序論}

\section{背景}
自動車のアクセルやブレーキペダル、ピアノやオルガンのペダルに代表されるように、我々は日常的に足による操作を行なっている。しかし、パーソナルコンピュータをデスクトップ上で作業をする際、我々は手を中心に操作を行い、足に操作が割り当てられることはない。足による操作を用いたコンピュータ向けインタフェースの研究は、1960年代から存在している\cite{1698228}が、現在は手による操作が中心である。しかし、Multitoe\cite{Augsten:2010:MHI:1866029.1866064}のようなタッチ認識を可能にした床面とのインタラクションや、手がふさがった状態におけるモバイル機器の操作\cite{Fan:2017:ESF:3123021.3123043, okumura_2011}など、新たに足によるインタラクションの研究は関心が高まっている。\par
その中でも、デスクトップでの特にコンピュータを使った作業中は両手をキーボードの操作に充てることが多い。そのため、足によるマウスのようなカーソルやポインタの操作を行う研究が盛んである。
%足によってマウスのような操作を行うことで、両手は他の操作を行ったまま、マウスポインタの制御が可能になる。
そのアプローチは、装置を足で動かす方法\cite{Pearson:1986:MMD:22627.22392, Pearson:1988:EET:49108.1046356}、可変摩擦機構を取り付けた靴\cite{Horodniczy:2017:FHE:3025453.3025625}があるが、これらは体の一部に装置を取り付けるあるいは大型な装置を用いるものであるため、衣服などに制限が生じたり持ち運びができないなどの制約が加わってしまう。
%このような物理的制約を不要にする研究は少ない\cite{velloso:hal-01599657}。



\section{目的・アプローチ}
\fixme{前節で述べた問題を解決するために、本研究の目的は、特別な装置を足に装着することなく、かつ簡単に、足を用いたコンピュータとのインタラクションを可能にすることである。}そのアプローチとして、机下に取り付けた装置から膝の動作を読み取るシステムを製作する。
%膝の動作を読み取るシステムの開発を行い、膝を使ってポインタ操作を行う手法を提案する。ユーザは机の下で膝を動かすことで、パーソナルコンピュータ上のポインタを操作することができる。システムのハードウェアには、距離センサを用いた。
%距離センサを用いたトラッキング技術には、例えばAIRBAR\cite{AIRBAR}やLumiwatch\cite{Xiao:2018:LOP:3173574.3173669}が挙げられるが、膝に適用した例はない。また、
%カメラを用いないことで、設置や構築が容易であるという利点があるために、カフェのテーブルや、ホテルのデスクのように様々な場所で設置、利用可能である。\\
\TODO{足から突然膝になっているのでつながりを設けたい}

\section{貢献}
本研究の貢献を以下に述べる。
\begin{itemize}
	\item 安易に設置可能かつ安価な装置で膝によるマウスポインタ操作を実現した。
	\item 距離センサを使い、読み取ったデータから画面上の座標にマウスポインタをマップするプロトタイプを開発した。
	\item プロトタイプを用いて、膝によるマウスポインタ操作をフィッツの法則に当てはめて評価し、\TODO{[結果]}
\end{itemize}
\section{本論文の構成}

 
\fixme{
	\begin{itemize}
		\item 本研究では膝によるマウスカーソル操作を調査する?
		\item 足を使ってみたい
		\item 足先の研究しかない、膝と組み合わせることで様々なインタラクションが可能になる
		\item 膝使ったものは少なく問題がある
		\item 膝を使ったことの理由→足の既存手法と組み合わせることができる、膝の可動域が広い(が先行研究が少ない
		\item いずれにせよ、関連研究が終わるまでに「膝によるマウスカーソル操作」という話に落とし込む
	\end{itemize}
}
