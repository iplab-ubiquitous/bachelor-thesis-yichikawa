\chapter{今後の展望}\chaLabel{discussion}
本章では,プロトタイプの既知の問題点に対する今後の展望について述べる.

今回のプロトタイプの設計は,直定規に距離センサを貼り付けたため,直線的な形状となった.しかし,膝を左右に傾ける時,膝の運動は直線運動ではなく円弧に近い運動をする.そのため膝がハードウェアの両端付近にある場合と中心付近にある場合とでは,ユーザが水平にカーソルを移動するとき,上下にカーソルがずれてしまう問題が発生する.加えて,ユーザの足の長さによって描く円弧の半径は異なると考える.このことから,膝の運動のユーザ間での違いについてより詳細な分析を行い,その結果からハードウェアの形状を再設計する必要がある.

%ハードウェアは直線形状ではなく円弧に近い形にする必要があり,さらにその形状も,ユーザによって簡単に変えることができるような設計が求められる.そのために,膝の運動のユーザ間での違いについてより詳細な分析も必要となる.

また,今回のプロトタイプでは距離センサを10個をそれぞれ30\si{mm}の間隔で配置した.この設計について,距離センサの数や配置の間隔を増減させた時にも同様なマルチディレクショナルポインティングタスクの実験を行うことで,最適な設計を導く.