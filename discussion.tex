\chapter{議論と今後の展望}\chaLabel{discussion}
本章では,現在分かっているプロトタイプの問題点と,実験の結果を受けて明らかになった膝の操作やプロトタイプの問題点について議論を行い,今後の展望について示す.
\section{ハードウェア設計}\secLabel{hd}
\subsection{膝の運動の特徴}
今回のプロトタイプの設計は,直定規に距離センサを貼り付けたため,直線的な形状となった.しかし,膝を左右に傾ける時,膝の運動は直線運動ではなく円弧に近い運動をする.そのため膝がハードウェアの両端付近にある場合と中心付近にある場合とでは,ユーザが水平にカーソルを移動したいと思っても,上下にカーソルがずれてしまう問題が発生する.加えて,ユーザの足の長さによって描く円弧の半径は異なると考える.このことから,ハードウェアは直線形状ではなく円弧に近い形にする必要があり,さらにその形状も,ユーザによって簡単に変えることができるような設計が求められる.そのために,膝の運動のユーザ間での違いについてより詳細な分析も必要となる.
\subsection{距離センサの個数}
今回のプロトタイプでは距離センサを10個をそれぞれ30\si{mm}の間隔で配置した.この設計について,距離センサの数や配置の間隔を増減させた時にも同様なマルチディレクショナルポインティングタスクの実験を行うことで,最適な設計を導きたい.

\section{エラー率とキャリブレーション}
\refCha{exp}で行なった実験の結果,膝による操作での選択ミス率は左膝で1.14\%,右膝で1.71\%であり,Horodniczyら\cite{Horodniczy:2017:FHE:3025453.3025625}やVellosoら\cite{velloso:hal-01599657}の手法の結果よりも低い値であった.原因として以下の3点が考えられる.

1点目はキャリブレーションを1セッションごと,計6回と多く行なっていたことである.このため,キャリブレーションの回数を減らすことでエラー率やスループットにどのような影響を与えるかを調査する必要がある.
2点目は膝はカーソル操作のみを行い,選択操作はキーボードのEnterキーで行なっていたことである.したがって,マウスの機能をすべて膝と足のみを使って実現した場合,エラー率が変化する可能性がある.
3点目はポインティングタスクの難易度である.今回の実験では,IDの最大は4.09[bit]であったが,最大5.67[bit]で行う実験も存在した\cite{Horodniczy:2017:FHE:3025453.3025625, velloso:hal-01599657}.したがって,よりIDの高い条件での実験を行う必要がある.

\section{操作性と疲労感の改善}
\refCha{system}にて,上下方向の膝の動かし方を疲労感を考慮したものとした.しかしアンケートでは,P1は太ももと足に疲労を感じると回答しており,今後動かし方をさらに見直し,足の操作の疲労感を改善する工夫が必要である.アンケートからはさらに,操作の使いやすさ,快適さ,スムーズさにも改善の余地があるとわかる.\refSec{hd}で述べたハードウェアの改善を行うことで,操作性についても改善を図る.
